\chapter{Summary and Conclusions}
With the upcoming end of Venus Express, earth-based centimeter and millimeter-wavelength observations of Venus are becoming more important. Knowledge of the absorption properties of sulfur dioxide will be important in analyzing data from these earth-based observations. Discussions of the model, remote sensing application, the use of the radiative transfer model, and suggestions for future work are presented below.
\section{Significant Results}
No data of SO$_2$'s millimeter-wavelength opacity has been previously taken, and thus, these reults are a significant increase in the understanding of the true absorption properties of SO$_2$.

The data taken for the millimeter-wavelength absorption of sulfur dioxide is able to verify formalism for sulfur dioxide opacity developed by Fahd and Steffes (1992) \cite{Fahd-1992} was accurate for both centimeter and millimeter-wavelengths. The model is able to fit 85.88\% of the laboratory data (centimeter-wavelength done by Steffes et al. 2014 \cite{Steffes-2014}) within $2\sigma$ uncertainty. The bounds verified by laboratory data are set to frequencies between 1-150 GHz, temperatures between 307-550 K, and concentrations between 0-100\% of SO$_2$/volume. The results of the model from input parameters greater then these absolute bounds are not guaranteed.
\section{Application to Venus Observations}
\section{Radiative Transfer Model}
\section{Suggestions for Future Work}
Venus RTM