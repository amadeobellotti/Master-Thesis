\section{Measurement Uncertainties}

There are five uncertainties for absorptivity measurements using the  centimeter and millimeter wavelength systems (Hanley \cite{Hanley-thesis}) at the Planetary Atmospheres Laboratory at The Georgia Institute of Technology: instrumentation errors and electrical noise ($Err_{inst}$), errors in dielectric matching ($Err_{diel}$), errors in transmissivity measurement ($Err_{trans}$), errors due to resonance asymmetry ($Err_{asym}$), and errors in measurement conditions ($Err_{cond}$) resulting from uncertainties in temperature, pressure, and mixing ratio. The term $Err$ is used for representing $2\sigma$ uncertainties.

Instrumental errors and electrical noise are due to the limited sensitivity of the electrical devices and their ability to accurately measure bandwidth ($BW_{measured}$) and the center frequency ($f_o$). Electrical noise arises from the limited-stability frequency references and the noise of the internal electronics. Electrical noise is uncorrelated and the best estimate of instrumental uncertainty is the variance of multiple measurements. The variance of the error estimate is given by the sample variance ($S^2_N$) weighted by the confidence coefficient ($B$) as
\begin{equation}\label{eq:sigman}
\sigma^2_N = B \frac{S^2_N}{N_{samples}}
\end{equation}

\noindent where $N_{samples}$ is the number of independent measurements of the sample. For the millimeter-wavelength system, five sets of independent measurements of each resonance are taken. A confidence coefficient ($B$) of 2.776 is used. This corresponds to the 95\% confidence interval ($2\sigma$). The center frequency standard deviation is very small and its effect on the uncertainty in $Q$ is negligible. Therefore, $S_N$ is the sample standard deviation of the bandwidth of the measurements.

The HP 8564E spectrum analyzer is used for measuring the resonances in the millmeter-wavelength system. It's manufacturer-specified instrumental uncertainties are the $3\sigma$ values \cite{Hewlett-Packard}. The $3\sigma$ standard deviation for the center frequency and bandwidth are estimated by 

\begin{equation}\label{eq:sigmao}
Err_o \leq \pm (f_o \times f_{ref\:acc} + 0.05 \times SPAN + 0.15 \times RBW +10 ) (Hz)
\end{equation}
\begin{equation}\label{eq:sigmabw}
Err_{BW} \leq \pm (BW_{measured} \times f_{ref\;acc} + 4 \times N_H +2 \times LSD ) (Hz)
\end{equation}

\noindent where $f_{ref\;acc}$ is given as
\begin{equation}\label{eq:frefacclong}
\begin{split}
f_{ref\;acc} = (aging \times {time\;since\;calibration}) + {inital\;achievable\;accuracy} \\
+ {\;temperature\;stability}
\end{split}
\end{equation}

\noindent and $f_o$, SPAN, RBW, $N_H$, and LSD are the center frequency, frequency span, resolution bandwidth, harmonic number, and least significant digit of the bandwidth measurement, respectively. LSD is calculated as 
\begin{equation}
LSD = 10^x
\end{equation} 
where x is the the smallest positive integer value of x such that SPAN $< 10^{x+4}$. For SPAN $\leq 2$ MHz$\times N_h$, Equation \ref{eq:sigmao} becomes 
\begin{equation}\label{eq:sigmaosmall}
Err_o \leq \pm (f_o \times f_{ref\:acc} + 0.01 \times SPAN + 0.15 \times RBW +10 ) (Hz)
\end{equation}
For the spectrum analyzer used, $f_{ref\;acc}$ reduces to

\begin{equation}\label{eq:frefacc}
f_{ref\;acc} = (10^{-7} \times {years\;since\;calibrated}) + 3.2\times 10^{-8}
\end{equation}

The worst case scenario is used to transform the uncertainty in center frequency and bandwidth for both loaded and dielectrically matched measurements into an uncertainty in absorptivity as described in DeBoer and Steffes \cite{DeBoer-Steffes}.

\begin{equation}
Err^2_\Psi = \langle {F_l^2}\rangle + \langle {F_m^2}\rangle -\langle {F_l F_m}\rangle
\end{equation}

\noindent where
\begin{equation}
\langle {F_i^2}\rangle = \frac{\Upsilon_i^2}{f_{oi}^2}
\left[ \frac{Err_o^2}{Q_l^2} + Err_{BW}^2 + Err_{Ni}^2 + \frac{2Err_o Err_{BW}}{Q_i} \right], i= l,m
\end{equation}
\begin{equation}
\langle {F_l F_m}\rangle = -\frac{\Upsilon_l \Upsilon_m}{f_{ol} f_{om}}
\left[ \frac{Err_o^2}{Q_i Q_m} + Err_{BW}^2 + \frac{Err_o Err_{BW}}{Q_l} + \frac{Err_o Err_{BW}}{Q_m}\right]
\end{equation}
\begin{equation}
Q_i = \frac{f_{oi}}{f_{BWi}}, i = l,m
\end{equation}
\begin{equation}
\Upsilon_i = 1- \sqrt{t}, i = l,m
\end{equation}
where $l,m$ denote loaded and dielectrically matched cases respectively and $f_{ol,om}$ and $f_{BWl,BWm}$ represent center frequency and bandwidth of loaded and dielectrically matched cases respectively. The $2\sigma$ uncertainty of the measured gas absorption due to instrumental errors and electrical noise is given by
\begin{equation}
Err_{inst} = \pm \frac{8.686\pi}{\lambda}Err_\Psi\;(dB/km)
\end{equation}
where $\lambda$ is the wavelength in km. 

Errors in dielectric matching occur when the center frequency of the matched measurements are not precisely aligned with the center frequency of the loaded measurement. Since the Q of the resonator can vary slightly, this causes an uncertainty in the Q of the matched measurement at the true center frequency of the loaded measurement. The method used to calculate the magnitude of this effect is similar to Hanley \cite{Hanley-thesis}. While this error is the smallest due to the high precision of the software controlled matching, it is important to calculate and account for. The magnitude of this effect is calculated by comparing the Q of the three vacuum measurements to that of the dielectric matched measurements

\begin{equation}
\left(\frac{dQ}{df} \right)_i = \left|\frac{Q_{vac,i} - Q_{matched,i}}{f_{vac,i} - f_{matched,i}} \right| \textnormal{ for } i = 1,2,3
\end{equation}

The maximum of the three values is used to calculate a $dQ$ value

\begin{equation}
dQ = \left(\frac{dQ}{df} \right)_{max} \times |f_{loaded} - f_{matched}|
\end{equation}
where $f_{loaded}$ and $ f_{matched}$ are the center frequencies of the resonances under loaded and matched conditions. The error in absorptivity due to imperfect dielectric matching is then computed by propagating $\pm dQ$ through Equation \ref{eq:alphamatch}.
% \times \left| \left( \frac{1-\sqrt{t_{loaded}}}{Q^m_{loaded}} - \frac{1-\sqrt{t_{matched}}}{Q^m_{matched} + dQ} \right) \left|
\begin{equation}
\begin{split}
Err_{diel} &= \frac{8.686 \pi}{\lambda} 
\\ &\times \left| \left( \frac{1-\sqrt{t_{loaded}}}{Q^m_{loaded}} - \frac{1-\sqrt{t_{matched}}}{Q^m_{matched} + dQ} \right) - \left( \frac{1-\sqrt{t_{loaded}}}{Q^m_{loaded}} - \frac{1-\sqrt{t_{matched}}}{Q^m_{matched} - dQ} \right) \right|\\
 &(dB/km)
\end{split}
\end{equation}

Transmissivity errors are due to the uncertainties in the measurement amplitude. This is caused by variations in gains of losses of the millimeter-wavelength instruments (signal generators and spectrum analyzer), cables, adapters, and waveguides used in this system. This is done by taking multiple test measurements of signal loss through the system without the FPR and finding the standard deviation ($S_N$) of the signal loss and weighing it by its confidence coefficient
\begin{equation}
Err_{msl} = \frac{4.303}{\sqrt{3}}S_N
\end{equation}

For the millimeter-wavelength system, the signal level measurements involve sampling the RF power with a WR-10 20 dB directional coupler to feed the harmonic mixer for down-conversion and detection. While this ensures that the input to the harmonic mixer does not exceed its maximum allowed input power of -10 dBm, the WR-10 20dB directional coupler does not uniformly sample the input signal throughout the entire frequency range. To compensate for this, an additional 1.5 dB uncertainty is added to insertion loss error. The signal generator has a temperature stability of 1 dB/$10^\circ$ C, but an internal temperature equilibrium is reached after two hours \cite{Hewlett-Packard}. Since the measurements units are stored at a constant temperature this uncertainty can be disregarded. The total uncertainty in insertion loss for the millimeter-wavelength system is calculated by
\begin{equation}
Err_{ins\;loss} = Err_{msl} +1.5\;(dB)
\end{equation}

The error in insertion loss is used to compute the transmissivity error
\begin{equation}
Err_{t,i} = \frac{1}{2} ( 10^{-S_i - Err_{ins\;loss}} - 10^{-S_i + Err_{ins\;loss}}) , i=l,m
\end{equation}
where l, m are the loaded and matched cases, respectively, and S is the insertion loss of the resonator. This is used to compute the $2\sigma$ uncertainties in opacity and is expressed as
\begin{equation}
\begin{split}
Err_{trans} &= \frac{8.686 \pi}{2\lambda}\\ 
 &\times \left| \left( \frac{\sqrt{t_l + Err_{t,l}}- \sqrt{t_l - Err_{t,l}}}{Q^m_{loaded}} - \frac{\sqrt{t_m - Err_{t,m}}- \sqrt{t_m + Err_{t,m}}}{Q^m_{matched}} \right) \right|\\
 &(dB/km).
\end{split}
\end{equation}

Errors from asymmetry are due to the asymmetric nature of the resonances. These are more prominent at low temperatures and short wavelengths. Errors due to the asymmetry result from the disproportionate asymmetric broadening of the loaded measurements compared to the matched measurements. Equivalent full bandwidths based on assuming symmetry of the high and low sides of the resonances are calculated as
\begin{equation}
BW_{high} = 2 \times (f_{high} - f_{center})
\end{equation}
\begin{equation}
BW_{low} = 2 \times (f_{center} - f_{low})
\end{equation}
where $BW_{high}, BW_{low}, f_{high}, f_{center}$, and $f_{low}$ are the high bandwidth, low bandwidth, higher frequency half power point, center frequency, and lower frequency half power point, respectively. For a perfectly symmetric resonance, $BW_{high} = BW_{low}$. The difference between the opacities calculated using $BW_{high}$ and $BW_{low}$ is defined as $Err_{asym}$ and is calculated by
\begin{equation}
\begin{split}
Err_{asym} &= \frac{8.686 \pi}{\lambda} 
\\ &\times \left| \left( \frac{1-\sqrt{t_{loaded}}}{Q^m_{loaded,high}} - \frac{1-\sqrt{t_{matched}}}{Q^m_{matched,high}} \right) - \left( \frac{1-\sqrt{t_{loaded}}}{Q^m_{loaded,low}} - \frac{1-\sqrt{t_{matched}}}{Q^m_{matched,low}} \right) \right|\\
 &(dB/km)
\end{split}
\end{equation}
where $Q^m_{matched,high/low}$ and $Q^m_{loaded,high/low}$ are the measured Q's evaluated using the high and low bandwidths for loaded and matched cases. 

The uncertainties in measured temperature, pressure, and concentration in the millimeter-wavelength system contribute to the total uncertainty due to the measurement conditions ($Err_{cond}$). While uncertainties in measurement conditions do not directly affect the measurements of millimeter-wavelength absorptivity, they still need to be accounted for when evaluating the opacity formalisms. It is computed by
\begin{equation}
Err_{cond} = \sqrt{Err_{temp}^2 + Err_{p}^2 + Err_{c}^2 } (dB/km)
\end{equation}
with $Err_{temp}$, $Err_{p} $, and $ Err_{c}$ representing the 2$\sigma$ uncertainties  in temperature, pressure, and concentration (or mole fraction) respectively. Each of these are calculated by taking the maximum modeled opacity with each uncertainty minus the minimum modeled opacity and halving the difference. 

Temperature was measured using a T type thermocoupler along with a Wavetek 23XT voltmeter. The voltmeter has a temperature accuracy of $\pm (1\% + 2^\circ C)$. Since the voltmeter has a cold compensation circuitry it is unnecessary to correct for ambient temperature. The temperature inside the test vessel is stable enough that it does not drift a significant amount during the hour it takes to run a test. The uncertainty in temperature is calculated by
\begin{equation}
T = T_{read} \pm ( T_{read} \times 1\% + 2)
\end{equation}
where $T_{read}$ is the temperature (in $^\circ$C) displayed by the Wavetek 23XT voltmeter.

Pressure was measured using an Omega DPG-7000 which has an accuracy of $\pm 0.05\%$FS (full scale). Since this pressure gauge measures pressure relative to ambient it is necessary to take a measurement before and after each test to ensure that the ambient pressure did not change significantly during the test. 
The average change in pressure during a test was at most 2 mbar. The cause of this change was identified as a change in ambient pressure during the test. Since the Omega DPG-7000 is a relative pressure gauge it was necessary to track ambient pressure. 
A vacuum was ensured by comparing the Omega DPG-7000 reading to that of an absolute pressure gauge (Druck DPI 104). The Druck has an accuracy of $\pm 0.05\%$FS as well as a resolution of $\pm 1$ mbar. The uncertainty in pressure reading is calculated by
\begin{equation}
P = P_{read} \pm ( P_{FS} \times .05\% + 3)
\end{equation}
where $P_{FS}$ is the full scale pressure of the Omega DPG-7000 (3.08 bars).

Since $Err_{cond}$ is dependent on the opacity model, this uncertainty is maintained separately from $Err_{tot}$. Thus the total 95\% confidence for the measurement uncertainty is expressed in dB/km as per Hanley \cite{Hanley-thesis}
\begin{equation}
Err_{tot} = \sqrt{Err_n^2 + Err_{diel}^2 + Err_{trans}^2 + Err_{asym}^2} \;(dB/km).
\end{equation}
