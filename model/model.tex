\chapter{Model Fitting and Modifications}
In total, 36 data sets were taken at two temperatures (12 at \textasciitilde 308 K and 24 at \textasciitilde 343 K) at 2-4 mm-wavelength. This, along with data taken at the centimeter-wavelength by Steffes et al. 2014 \cite{Steffes-2014}(10 data sets at \textasciitilde 435 K, 10 data sets at \textasciitilde 490 K, and 5 data sets at \textasciitilde 550 K), were used in finding the best-fit model.

Before creating a new formalism for the absorption of SO$_2$ in a CO$_2$ atmosphere, analysis of previous models was conducted. The Van Vleck and Weisskopf Model (VVW) used by Fahd and Steffes \cite{Fahd-thesis} with the new JPL rotational line catalog (Pickett, et al. 1998 \cite{Pickett-1998}) was found to fit 85.88\% of all 500 within $2\sigma$ uncertainty. Consideration of the model analysis process and the final model are presented.
\section{Measurement Uncertainties}

There are five uncertainties for any absorptivity measurements using the centimeter and millimeter wavelength system: instrumentation errors and electrical noise ($Err_{inst}$), errors in dielectric matching ($Err_{diel}$), errors in transitivity measurement ($Err_{trans}$), errors due to resonance asymmetry ($Err_{asym}$), and errors in measurement conditions ($Err_{cond}$) resulting from uncertainties in temperature, pressure, and mixing ratio. The term $Err$ is used for representing uncertainties instead of the more frequently used $\sigma$ to avoid confusion between $1\sigma$,$2\sigma$, and $3\sigma$ uncertainties.

Instrumental errors and electrical noise are caused due to the sensitivity of the electrical devices and their ability to accurately measure bandwidth ($BW_{measured}$) and the center frequency ($f_o$). Electrical noise arises from the frequency references and the noise of the internal electronics. Since electrical noise is uncorrelated, it's best estimate of the uncertainty is the mean of multiple measurements. The variance of the best error estimate is given by the sample variance ($S^2_N$) weighted by the confidence coefficient ($B$) as
\begin{equation}\label{eq:sigman}
\sigma^2_N = B \frac{S^2_N}{N_{samples}}
\end{equation}

\noindent where $N_{samples}$ is the number of independent measurements of the sample. For the millimeter-wavelength system, five sets of independent measurements of each resonance are taken. A confidence coefficient ($B$) of 2.776 is used. This corresponds to the 95\% confidence interval. The center frequency standard deviation is very small and its effect on the uncertainty in $Q$ is negligible. Therefore, $S_N$ i the sample standard deviation of the bandwidth of the measurements.

The HP 8564E spectrum analyzer is used for measuring the resonances in the millmeter-wavelength system. It's manufacturer specified instrumental uncertainties are the $3\sigma$ values \cite{Hewlett-Packard}. The $3\sigma$ standard deviation for the center frequency and bandwidth are estimated by 

\begin{equation}\label{eq:sigmao}
Err_o \leq \pm (f_o \times f_{ref\:acc} + 0.05 \times SPAN + 0.15 \times RBW +10 ) (Hz)
\end{equation}
\begin{equation}\label{eq:sigmabw}
Err_{BW} \leq \pm (BW_{measured} \times f_{ref\;acc} + 4 \times N_H +2 \times LSD ) (Hz)
\end{equation}

\noindent where $f_{ref\;acc}$ is given as

\begin{equation}\label{eq:frefacclong}
\begin{split}
f_{ref\;acc} = (aging \times {time\;since\;calibration}) + {inital\;achievable\;accuracy} \\
+ {temperature\;stability}
\end{split}
\end{equation}

\noindent and $f_o$, SPAN, RBW, $N_H$, and LSD are the center frequency, frequency span, resolution bandwidth, harmonic number, and least significant digit of the bandwidth measurement, respectively. LSD is calculated as $LSD = 10^x$ for the smallest positive integer value of x such that SPAN $< 10^{x+4}$. For SPAN $\leq 2$ MHz$\times N_h$, the SPAN multiplication factor of 0.05 is replaced with 0.01. For the spectrum analyzer used, $f_{ref\;acc}$ reduces to

\begin{equation}\label{eq:frefacc}
f_{ref\;acc} = (10^{-7} \times {years\;since\;calibrated}) + 3.2\times 10^{-8}
\end{equation}

The worst case scenario is used to transform the uncertainty in center frequency and bandwidth for both loaded and dielectrically matched measurements into an uncertainty in absorptivity as described in DeBoer and Steffes \cite{DeBoer-Steffes}.

\begin{equation}
Err^2_\Psi = \langle {F_l^2}\rangle + \langle {F_m^2}\rangle -\langle {F_l F_m}\rangle
\end{equation}

\noindent where
\begin{equation}
\langle {F_i^2}\rangle = \frac{\Upsilon_i^2}{f_{oi}^2}
\left[ \frac{Err_o^2}{Q_l^2} + Err_{BW}^2 + Err_{Ni}^2 + \frac{2Err_o Err_{BW}}{Q_i} \right], i= l,m
\end{equation}
\begin{equation}
\langle {F_l F_m}\rangle = -\frac{\Upsilon_l \Upsilon_m}{f_{ol} f_{om}}
\left[ \frac{Err_o^2}{Q_i Q_m} + Err_{BW}^2 + \frac{Err_o Err_{BW}}{Q_l} + \frac{Err_o Err_{BW}}{Q_m}\right]
\end{equation}
\begin{equation}
Q_i = \frac{f_{oi}}{f_{BWi}}, i = l,m
\end{equation}
\begin{equation}
\Upsilon_i = 1- \sqrt{t}, i = l,m
\end{equation}
where $l,m$ denote loaded and dielectrically matched cases, respectively and $f_{ol,om}$ and $f_{BWl,BWm}$ represent center frequency and bandwidth of loaded and dielectrically matched cases respectively. The $2\sigma$ uncertainty of the measured gas absorption due to instrumental errors and electrical noise is given by
\begin{equation}
Err_{inst} = \pm \frac{8.686\pi}{\lambda}Err_\Psi\;(dB/km)
\end{equation}
where $\lambda$ is the wavelength in km. 

Errors in dielectric matching occur when the when the center frequency of the matched measurements are not precisely aligned with the center frequency of the loaded measurement. Since the Q of the resonator can vary slightly this causes an uncertainty in the Q of the matched measurement at the true center frequency of the loaded measurement. The method used to calculate the magnitude of this effect is similar to Devaraj \cite{Devaraj-thesis}. While this error is the most trivial due to the high precision of the software controlled matching it is important to calculate and account for.The magnitude of this effect is calculated by comparing the Q of the three vacuum measurements to that of the dielectric matched measurements

\begin{equation}
\left(\frac{dQ}{df} \right)_i = \left|\frac{Q_{vac,i} - Q_{matched,i}}{f_{vac,i} - f_{matched,i}} \right| \textnormal{ for } i = 1,2,3
\end{equation}

The maximum of the three values is used to calculate a $dQ$ value

\begin{equation}
dQ = \left(\frac{dQ}{df} \right)_{max} \times |f_{loaded} - f_{matched}|
\end{equation}
where $f_{loaded}$ and $ f_{matched}$ are the center frequencies of the resonances under loaded and matched conditions. The error in absorbtivity due to imperfect dielectric matching is then computed by propagating $\pm dQ$ through Equation \ref{eq:alphamatch}.
% \times \left| \left( \frac{1-\sqrt{t_{loaded}}}{Q^m_{loaded}} - \frac{1-\sqrt{t_{matched}}}{Q^m_{matched} + dQ} \right) \left|
\begin{equation}
\begin{split}
Err_{diel} &= \frac{8.686 \pi}{\lambda} 
\\ &\times \left| \left( \frac{1-\sqrt{t_{loaded}}}{Q^m_{loaded}} - \frac{1-\sqrt{t_{matched}}}{Q^m_{matched} + dQ} \right) - \left( \frac{1-\sqrt{t_{loaded}}}{Q^m_{loaded}} - \frac{1-\sqrt{t_{matched}}}{Q^m_{matched} - dQ} \right) \right|\\
 &(dB/km)
\end{split}
\end{equation}

Transmissivity errors are due to the uncertainties in the measurement amplitude. This is caused by loss in the millimeter-wavelength instruments (signal generators and spectrum analyzer), cables, adapters, and waveguides used in this system. Measuring this is done taking multiple tests of the system without the FPR and finding the standard deviation ($S_N$) and weighing it by it's confidence coefficient
\begin{equation}
Err_{msl} = \frac{4.303}{\sqrt{3}}S_N
\end{equation}

For the millimeter-wavelength system, the signal level measurements involve sampling the RF power with a WR-10 20 dB directional coupler to feed the harmonic mixer for down-conversion and detection. While this ensures that the input to the harmonic mixer does not exceed its maximum allowed input power of -10 dBm, the WR-10 does not uniformly sample the input signal throughout the entire frequency range. To compensate for this an aditional 1.5 dB uncertainty is added to insertion loss error. The signal generator has a temperature stability of 1 dB/$10^\circ$ C, but an internal temperature equilibrium is reached after two hours \cite{Hewlett-Packard}. Since the measurements units are stored at a constant temperature this uncertainty can be disregarded. The total uncertainty in insertion loss for the millimeter-wavelength system is calculated by
\begin{equation}
Err_{ins\;loss} = Err_{msl} +1.5\;(dB)
\end{equation}

The error in insertion loss is used to compute the transissivity error
\begin{equation}
Err_{t,i} = \frac{1}{2} ( 10^{-S_i - Err_{ins\;loss}} - 10^{-S_i + Err_{ins\;loss}}) , i=l,m
\end{equation}
where l,m are the loaded and matched cases, respecivey, and S is the insertion loss of the resonator. This is used to compute the $2\sigma$ uncertainties in opacity and is expressed as
\begin{equation}
\begin{split}
Err_{trans} &= \frac{8.686 \pi}{2\lambda}\\ 
 &\times \left| \left( \frac{\sqrt{t_l + Err_{t,l}}- \sqrt{t_l - Err_{t,l}}}{Q^m_{loaded}} - \frac{\sqrt{t_m - Err_{t,m}}- \sqrt{t_m + Err_{t,m}}}{Q^m_{matched}} \right) \right|\\
 &(dB/km).
\end{split}
\end{equation}

Errors from asymmetry are due to the asymmetric nature of the resonances. These are more prominent at low temperatures and short wavelength. Errors due the asymmetry results from the disproportionate asymmetric broadening of the loaded measurements compared to the matched measurements. Equivalent full bandwidths based on assuming symmetry of the high and low sides of the resonances are calculated as
\begin{equation}
BW_{high} = 2 \times (f_{high} - f_{center})
\end{equation}
\begin{equation}
BW_{low} = 2 \times (f_{center} - f_{low})
\end{equation}
where $BW_{high}, BW_{low}, f_{high}, f_{center}$, and $f_{low}$ are the high bandwidth, low bandwidth, higher frequency half power point, center frequency, and lower frequency half power point, respectively. The difference between the opacities calculated using $BW_{high}$ and $BW_{low}$ is defined as $Err_{asym}$ and is calculated by
\begin{equation}
\begin{split}
Err_{asym} &= \frac{8.686 \pi}{\lambda} 
\\ &\times \left| \left( \frac{1-\sqrt{t_{loaded}}}{Q^m_{loaded,high}} - \frac{1-\sqrt{t_{matched}}}{Q^m_{matched,high}} \right) - \left( \frac{1-\sqrt{t_{loaded}}}{Q^m_{loaded,low}} - \frac{1-\sqrt{t_{matched}}}{Q^m_{matched,low}} \right) \right|\\
 &(dB/km)
\end{split}
\end{equation}
Where $Q^m_{matched,high/low}$ and $Q^m_{loaded,high/low}$ are the measured Qs evaluated using the high and low bandwidths for loaded and matched cases. 

The measured uncertainties in temperature, pressure, and concentration in the millimeter-wavelength system contribute to the total uncertainties due to the measurement conditions ($Err_{cond}$). While this does not affect the measurements it still needs to be accounted for during the creation of the models for opacity based on experimental data. It is computed by
\begin{equation}
Err_{cond} = \sqrt{Err_{temp}^2 + Err_{p}^2 + Err_{c}^2 + Err_{leak}^2} (dB/km)
\end{equation}
with $Err_{temp}^2, Err_{p}^2 $, $ Err_{c}^2$, and $Err_{leak}$ representing the 2$\sigma$ uncertainties in the proposed opacity model corresponding to the uncertainties in temperature, pressure, concentration, and leakage. 

Measuring temperature was done using a T type thermocoupler along with a Wavetek 23XT voltmeter. The voltmeter has a temperature accuracy of $\pm (1\% + 2^\circ C)$. Since the voltmeter has a Cold Compensation circuitry it is unnecessary to modify the temperature read from ambient. Also since a test takes an hour to run the temperature drift is insignificant. The uncertainty in temperature reading is calculated by
\begin{equation}
T = T_{read} \pm ( T_{read} \times 1\% + 2)
\end{equation}
Where $T_{read}$ is the temperature readout from the Voltmeter.

Pressure was measured using an Omega DPG-7000 which has an accuracy of $\pm 0.05\%$FS. Since this pressure gauge measures pressure relative to ambient it is necessary to take a measurement before and after each test. The average change in pressure during a test was at most 2 mbar. The way a vacuum was measured was by comparing the Omega DPG-7000 reading to that of an absolute pressure gauge (Druck DPI 104). The Druck has an accuracy of $\pm 0.05\%$FS as well but a resolution of $\pm 1$ mbar. The uncertainty in pressure reading is calculated by
\begin{equation}
P = P_{read} \pm ( P_{FS} \times .05\% + 3)
\end{equation}
Where $P_{FS}$ is the Full Scale pressure of the Omega DPG-7000.

Since $Err_{cond}$ is dependent on the opacity model, this uncertainty is maintained separately from $Err_{tot}$. Thus the total 95\% confidence for the measurement uncertainty is expressed in dB\/km as per Hanley \cite{Hanley-thesis}
\begin{equation}
Err_{tot} = \sqrt{Err_n^2 + Err_{diel}^2 + Err_{trans}^2 + Err_{asym}^2} \;(dB/km).
\end{equation}


\section{Model Analysis Process}

The models used in this comparison are the Van Vleck-Weisshopf model (using coefficients from Fahd and Steffes \cite{Fahd-1991}) and the Ben-Reuven model as calculated by Suleiman et al. \cite{Suleiman-1996}. The comparison of these models are done using a L$_2$ norm analysis. 

The following compliance function was used to calculate the number of data points that each model encompassed,
\begin{equation}
\textbf{1}_{model}(\alpha) = \left\{
     \begin{array}{lr}
       1 & : |\alpha_{measured} - \alpha_{model}| \leq \sqrt{Err_{tot}^2 + Err_{cond}^2 }\\
       0 & : |\alpha_{measured} - \alpha_{model}| > \sqrt{Err_{tot}^2 + Err_{cond}^2 }
     \end{array}
   \right.
\end{equation}
where $\textbf{1}_{model}$ is the compliance function for each model, $\alpha_{measured}$, $\alpha_{model}$ is the measured absorption and the calculated absorption respectively. $Err_{tot}$ and $Err_{cond}$ are the systematic and conditional errors as described previously. The percentage of data points that each model encompasses can be calculated using,
\begin{equation}
Per_{model} = \frac{\sum_{i=1}^N \textbf{1}_{model}(\alpha_i)}{N}\times 100\%
\end{equation}
where $Per_{model}$ is the percentage of data points that the model fits and $N$ is the total number of data points. The final results are summarized in Table \ref{tab:model-comp}.

\begin{sidewaystable}[p]

\caption{The percentage of the measured data points within $2\sigma$ uncertainty of the different models}
  \begin{tabular}{l | c c | r}
  \hline
  \hline
  SO$_2$ opacity model & Centimeter-Wavelength (1-8 GHz) & Millimeter-Wavelength (80-150 GHz) &Total\\
  \hline
  Fahd and Steffes (1992)	 & 82.97\%	& 	92.86\%& 85.88\%\\
  Sulieman (1997)& 62.98\%	&	88.10\%&70.37\%\\
  \hline
  \hline
  \end{tabular}
  \label{tab:model-comp}
\end{sidewaystable}

 
\clearpage
\section{Experimental Results}
High accuracy laboratory measurements of the temperature and pressure dependence of the millimeter-wavelength absorption of gaseous SO$_2$ in a CO$_2$ atmosphere have been conducted at 308K and 345 K and at pressures from 30 mbar to 3 bars for wavelengths between 2-4 millimeters. The following plots show the results of these absorptivity measurements with the accompanying 2$\sigma$ uncertainties. For comparison purposes these plots also show two known formalisms of SO$_2$'s absorptivity. One developed by Suleiman et al. 1997 \cite{Suleiman-thesis} and the second by Fahd and Steffes \cite{Fahd-thesis} but using the new JPL line catalog \cite{Pickett-1998}. 

\subsection{Accuracy of Constituents}

It is necessary to ensure that the gases used in each experiment are correctly characterized. Initially the bottle of SO$_2$ was assumed to consist of 100\% SO$_2$. The bottle was sent to Airgas$_\circledR$ for analysis. It was concluded that the SO$_2$ bottle used was actually comprised of 84.7\% SO$_2$ and 15.3\% N$_2$. Since N$_2$ has no absorptivity at centimeter and millimeter-wavelengths it can be safely disregarded. Thus, the SO$_2$  and CO$_2$ mole fractions do not add up to 100\% in the following plots. 


%Originally when the data was taken, the results fit the lineshapes of these two model's but the absorption calculated was much lower then predicted. After careful considerations of what could have caused this the SO$_2$ bottle was shipped back to Airgas\texttrademark for analysis. It was concluded that the SO$_2$ bottle used was comprised of 84.7\% SO$_2$ and 15.3\% N$_2$. Luckily N$_2$ has little to no absorptivity in the millimeter-wavelength domain so it is safely ignored.

%When this factor was then taken into account it became possible to compute both model's with the correct SO$_2$ mixing ratio. When this was done the data fit the model's lineshape and absorption. This is also the reason that the SO$_2$ mixing ratio and CO$_2$ mixing ratio do not add up to 100\%.

The CO$_2$ tank used was the same tank used in Steffes et al. \cite{Steffes-2015}. Since the CO$_2$ absorption measured in that paper matched the previously published formalism for opacity and refractive index of CO$_2$ \cite{Ho-1966}, it can be assumed that the tank contained pure CO$_2$.

%\subsection{Data}

%The systems had a tenancy to lose resonate peaks when the temperature changed. This causes the variable data points in different plots. Data was also removed if it was not physically possible (negative or huge, $>1000$dB/km, opacity) and if the error bars were too large (encompassing 50 orders of magnitude). These were caused by the system either not finding the expected resonant peaks and/or (in the case of the F-Band system) interference due to outside sources. While both of these issues were minimized as much as possible they still occurred. 


\begin{figure}[p]
 \centering 
\includegraphics[width=0.7\textwidth]{./model/results/{{308.75K-0.03bar-25so2-modelComparison}}} 
 \caption{Opacity data  using the 2-3 mm-wavelength system for a mixture of SO$_2$ = 84.7\% , CO$_2$ = 0\% at a pressure of 0.030 bar and a temperature of 308.8 K compared to various models}
 \end{figure}

\begin{figure}[p]
 \centering 
\includegraphics[width=0.7\textwidth]{./model/results/{{308.55K-0.97bar-25so2-modelComparison}}} 
 \caption{Opacity data  using the 2-3 mm-wavelength system for a mixture of SO$_2$ = 2.6\% , CO$_2$ = 96.9\% at a pressure of 0.970 bar and a temperature of 308.5 K compared to various models}
 \end{figure}

\begin{figure}[p]
 \centering 
\includegraphics[width=0.7\textwidth]{./model/results/{{308.65K-1.995bar-25so2-modelComparison}}} 
 \caption{Opacity data  using the 2-3 mm-wavelength system for a mixture of SO$_2$ = 1.3\% , CO$_2$ = 98.5\% at a pressure of 1.995 bar and a temperature of 308.6 K compared to various models}
 \end{figure}

\begin{figure}[p]
 \centering 
\includegraphics[width=0.7\textwidth]{./model/results/{{307.55K-0.116bar-98so2-modelComparison}}} 
 \caption{Opacity data  using the 3-4 mm-wavelength system for a mixture of SO$_2$ = 84.7\% , CO$_2$ = 0\% at a pressure of 0.116 bar and a temperature of 307.5 K compared to various models}
 \end{figure}

\begin{figure}[p]
 \centering 
\includegraphics[width=0.7\textwidth]{./model/results/{{307.25K-0.9429bar-98so2-modelComparison}}} 
 \caption{Opacity data  using the 2.7-4 mm-wavelength system for a mixture of SO$_2$ = 10.4\% , CO$_2$ = 87.7\% at a pressure of 0.943 bar and a temperature of 307.2 K compared to various models}
 \end{figure}

\begin{figure}[p]
 \centering 
\includegraphics[width=0.7\textwidth]{./model/results/{{307.25K-1.9869bar-98so2-modelComparison}}} 
 \caption{Opacity data  using the 2.7-4 mm-wavelength system for a mixture of SO$_2$ = 4.9\% and CO$_2$ = 94.2\% at a pressure of 1.987 bar and a temperature of 307.2 K compared to various models}
 \end{figure}

\begin{figure}[p]
 \centering 
\includegraphics[width=0.7\textwidth]{./model/results/{{344.45K-0.09bar-76so2-modelComparison}}} 
 \caption{Opacity data  using the 2-3 mm-wavelength system for a mixture of SO$_2$ = 84.7\% and CO$_2$ = 0\% at a pressure of 0.090 bar and a temperature of 344.4 K compared to various models}
 \end{figure}

\begin{figure}[p]
 \centering 
\includegraphics[width=0.7\textwidth]{./model/results/{{344.65K-0.923bar-76so2-modelComparison}}} 
 \caption{Opacity data  using the 2-3 mm-wavelength system for a mixture of SO$_2$ = 8.3\% and CO$_2$ = 90.2\% at a pressure of 0.923 bar and a temperature of 344.6 K compared to various models}
 \end{figure}

\begin{figure}[p]
 \centering 
\includegraphics[width=0.7\textwidth]{./model/results/{{343.95K-1.967bar-76so2-modelComparison}}} 
 \caption{Opacity data  using the 2-3 mm-wavelength system for a mixture of SO$_2$ = 3.9\% and CO$_2$ = 95.4\% at a pressure of 1.967 bar and a temperature of 343.9 K compared to various models}
 \end{figure}

\begin{figure}[p]
 \centering 
\includegraphics[width=0.7\textwidth]{./model/results/{{344.35K-0.033bar-28so2-modelComparison}}} 
 \caption{Opacity data  using the 2-3 mm-wavelength system for a mixture of SO$_2$ = 84.7\% and CO$_2$ = 0\% at a pressure of 0.033 bar and a temperature of 344.3 K compared to various models}
 \end{figure}

\begin{figure}[p]
 \centering 
\includegraphics[width=0.7\textwidth]{./model/results/{{344.55K-0.944bar-28so2-modelComparison}}} 
 \caption{Opacity data  using the 2-3 mm-wavelength system for a mixture of SO$_2$ = 3\% and CO$_2$ = 96.5\% at a pressure of 0.944 bar and a temperature of 344.5 K compared to various models}
 \end{figure}

\begin{figure}[p]
 \centering 
\includegraphics[width=0.7\textwidth]{./model/results/{{344.45K-2.007bar-28so2-modelComparison}}} 
 \caption{Opacity data  using the 2-3 mm-wavelength system for a mixture of SO$_2$ = 1.4\% and CO$_2$ = 98.4\% at a pressure of 2.007 bar and a temperature of 344.4 K compared to various models}
 \end{figure}

\begin{figure}[p]
 \centering 
\includegraphics[width=0.7\textwidth]{./model/results/{{343.65K-0.101bar-86so2-modelComparison}}} 
 \caption{Opacity data  using the 2.7-4 mm-wavelength system for a mixture of SO$_2$ = 84.7\% and CO$_2$ = 0\% at a pressure of 0.101 bar and a temperature of 343.6 K compared to various models}
 \end{figure}

\begin{figure}[p]
 \centering 
\includegraphics[width=0.7\textwidth]{./model/results/{{343.25K-0.936bar-86so2-modelComparison}}} 
 \caption{Opacity data  using the 2.7-4 mm-wavelength system for a mixture of SO$_2$ = 9.1\% and CO$_2$ = 89.2\% at a pressure of 0.936 bar and a temperature of 343.2 K compared to various models}
 \end{figure}

\begin{figure}[p]
 \centering 
\includegraphics[width=0.7\textwidth]{./model/results/{{342.95K-2.016bar-86so2-modelComparison}}} 
 \caption{Opacity data  using the 2.7-4 mm-wavelength system for a mixture of SO$_2$ = 4.2\% and CO$_2$ = 95\% at a pressure of 2.016 bar and a temperature of 342.9 K compared to various models}
 \end{figure}

\begin{figure}[p]
 \centering 
\includegraphics[width=0.7\textwidth]{./model/results/{{343.15K-0.06bar-51so2-modelComparison}}} 
 \caption{Opacity data  using the 2.7-4 mm-wavelength system for a mixture of SO$_2$ = 84.7\% and CO$_2$ = 0\% at a pressure of 0.060 bar and a temperature of 343.1 K compared to various models}
 \end{figure}

\begin{figure}[p]
 \centering 
\includegraphics[width=0.7\textwidth]{./model/results/{{343.65K-0.927bar-51so2-modelComparison}}} 
 \caption{Opacity data  using the 2.7-4 mm-wavelength system for a mixture of SO$_2$ = 5.5\% and CO$_2$ = 93.5\% at a pressure of 0.927 bar and a temperature of 343.6 K compared to various models}
 \end{figure}

\begin{figure}[p]
 \centering 
\includegraphics[width=0.7\textwidth]{./model/results/{{343.95K-2.004bar-51so2-modelComparison}}} 
 \caption{Opacity data  using the 2.7-4 mm-wavelength system for a mixture of SO$_2$ = 2.5\% and CO$_2$ = 97\% at a pressure of 2.004 bar and a temperature of 343.9 K compared to various models}
 \end{figure}



\clearpage

\section{Suggested Model}
Results indicate that the models for the centimeter- and millimeter-wavelength opacity from SO$_2$ in a CO$_2$ atmosphere by Suleiman et al. (1996) and Fahd and Steffes (1992) are both valid over the entire centimeter-and millimeter-wavelength range under simulated conditions for the upper atmosphere of Venus. Based on the percentage of data consistent with the models, the suggested model is the Fahd and Steffes \cite{Fahd-thesis} model. This model employs the Van Vleck-Weisskopf lineshape, and was developed from measurements of SO$_2$/CO$_2$ mixtures conducted at room temperature. As per their paper, we employ only the rotational line catalog to compute opacity. (JPL spectral line catalog, Pickett et al., 1998) \cite{Pickett-1998}. While both models perform well, the Fahd and Steffes (1992) model appears to provide a slightly better fit to the overall data set. 

It should also be noted that because both models were derived from measurements conducted at pressures of 6 Bars or less, no allowance for the compressibility of CO$_2$ is included in these models. When performing the best-fit analysis, a correction factor for compressibility was computed and entered into the models (simply dividing the measured partial pressure of CO$_2$ by the compressibility, Z). 

