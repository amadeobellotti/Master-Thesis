\section{Preliminary Results}

Currently the millimeter-wavelength system is completely operational in the Planetary Atmospheres Laboratory at The Georgia Institute of Technology. Using this system, high precision measurements of SO$_2$'s millimeter-wavelength absorption have recently been completed as part of this work. A preliminary model of SO$_2$'s absorption properties is available from Suleiman's previous work on microwave laboratory measurements \cite{Suleiman-thesis}. Additionally measurements  of SO$_2$'s centimeter-wavelength absorption have recently been taken under deep Venus conditions \cite{so2-cent-lab} \cite{so2-cent-model}. 

\subsection{Millimeter-Wavelength Results}

Only one previous measurement of SO$_2$'s mm-wavelength opacity under Venus simulated conditions has been done (see Fahd et. al. 1991) \cite{fahd-so2} This measurement was done using only one frequency (94.1 GHz) in the mm-wavelength spectra.

In our experiment, eight different frequencies have been already tested using the millimeter-wavelength system measuring 100 mbar of SO$_2$ along with separate tests at 1 bar CO$_2$ combined with the SO$_2$  and 2 bar of CO$_2$ combined with the SO$_2$. This allows for a comparison with Fahd's model and with Suleiman's model at higher frequencies.  

\subsubsection{Absorption Model}

The goal of the laboratory measurements is to create a mathematical model that accurately estimates the opacity of sulfur dioxide in a carbon dioxide atmosphere under all possible conditions of temperature, pressure, concentration, and frequency (fTPC space). For the data fitting process we will use data taken with the FPR along with data from the Planetary Atmospheres Laboratory centimeter-wavelength system  \cite{so2-cent-lab} \cite{so2-cent-model} to create a model that best fits the fTPC space. 

Extrapolating models for SO$_2$'s absorption into the mm-wavelength allows for a good starting point in the model creation process. As visible in the following figures, the absorption model matches the same shape as previous models but is lower by about 20\%. Finding a unified model for SO$_2$'s absorption will compensate for this change.

Figures \ref{fig:so2-116}, \ref{fig:so2-943}, and \ref{fig:so2-1987} show the initial data taken for SO$_2$ opacity in the 3-4 mm-wavelength range. It is clear that previous absorption models work well in predicting the shape of the millimeter-wavelength absorption spectrum of SO$_2$.

\begin{figure}[H]
    \centering
	\includegraphics[width=0.8\textwidth]{./plots/35C_W_High/{0.116-modelComparison}.eps}
	\caption{Measured absorption spectrum given 116 mbar of SO$_2$ at 308K in the W-band range. Shown for comparison are models from Devaraj (2011), Suleiman et al (1996), and Fahd and Steffes (1992).}
    \label{fig:so2-116}
\end{figure}

\begin{figure}[H]
    \centering
	\includegraphics[width=0.8\textwidth]{./plots/35C_W_High/{0.943-modelComparison}.eps}
	\caption{Measured absorption spectrum given 116 mbar of SO$_2$ and 827 mbar of CO$_2$ at 308K in the W-band range.
	Shown for comparison are models from Devaraj (2011), Suleiman et al (1996), and Fahd and Steffes (1992).}
    \label{fig:so2-943}
\end{figure}

\begin{figure}[H]
    \centering
	\includegraphics[width=0.8\textwidth]{./plots/35C_W_High/{1.987-modelComparison}.eps}
	\caption{Measured absorption spectrum given 116 mbar of SO$_2$ and 1871 mbar of CO$_2$ at 308K in the W-band range.
	Shown for comparison are models from Devaraj (2011), Suleiman et al (1996), and Fahd and Steffes (1992).}
    \label{fig:so2-1987}
\end{figure}