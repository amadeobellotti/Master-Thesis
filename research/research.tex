\section{Proposed Research}
The objective of the proposed research is to advance the knowledge of the millimeter-wavelength properties of gaseous sulfur dioxide under Venus conditions. As part of the proposed research, extensive laboratory measurements of the W-band and F-band properties of sulfur dioxide under simulated upper Venus atmosphere are ongoing. 

Upon completion of the laboratory measurements, efforts toward developing a unified model to estimate the centimeter and millimeter-wavelength opacity spectra of sulfur dioxide at various pressures, temperatures, and mixing ratios will be developed. 

\subsection{Significance of this Work}
The primary objective of this millimeter-wavelength research is to better understand properties of gaseous sulfur dioxide under Venus conditions. The laboratory measurements will help create a model that accurately estimates the opacity of sulfur dioxide in a carbon dioxide atmosphere at any temperature or pressure. The new model will provide a unified opacity model for sulfur dioxide at the centimeter and millimeter-wavelengths. This model can be used for accurate retrievals of sulfur dioxide from ground-based and spacecraft-based radio observations. 
\subsection{Planned Work}

Two major activities support this thesis: (1) laboratory measurements of 2-4 millimeter-wavelength properties of sulfur dioxide in a carbon dioxide atmosphere at two temperatures, 308 K and 348 K with pressures up to two bars and (2) the development of a model that best estimates sulfur dioxide's millimeter-wavelength absorption properties in Venus's upper troposphere. Both will be completed by Fall Semester 2014.


\subsubsection{Laboratory Measurements}

The ongoing millimeter-wavelength measurements  focus on characterizing the opacity of sulfur dioxide in a carbon dioxide atmosphere temperatures of 308 K and 348 K. These will be the first precision measurements done of sulfur dioxide's absorption properties at millimeter-wavelengths. Previous work by Fahd \cite{fahd-so2} included a measurement done at 94.1 GHz but it did not utilize the high precision measurement tools available for this thesis.  

The testing protocol involve laboratory measurements of the opacity of only sulfur dioxide as well as a mixture of sulfur dioxide and carbon dioxide at pressures of 1 bar and 2 bar. Table \ref{tab:testmatrix} shows a testing matrix of the tests to be done. Varying the sulfur dioxide abundance as well as the temperature allows for an accurate model to be created and extrapolated to other pressure and temperature combinations. 

\begin{table}[H]
    \centering
	\begin{tabular}{|c|c|c|c|c|}
	\hline
	Test Number & Gas under test & Pressure & Temperature & Subsystem\\
	\hline
	\hline
	1 & SO$_2$ & 100 mbar & 308 K & W-band\\
	2 & SO$_2$ & 100 mbar & 343 K & W-band\\
	3 & SO$_2$ & 60 mbar & 343 K & W-band\\
	4 & SO$_2$ & 100 mbar & 343 K & F-band\\
	5 & SO$_2$ & 30 mbar & 343 K & F-band\\
	6 & SO$_2$ & 30 mbar & 308 K & F-band\\
	\hline
	\end{tabular}
	\caption{Testing matrix for SO$_2$'s microwave absorption properties at 2-4 millimeter-wavelength.}
    \label{tab:testmatrix}
\end{table}


\subsubsection{Model Development}

Completed and planned centimeter- \cite{so2-cent-lab} \cite{so2-cent-model} and millimeter-wavelength measurements of the opacity of sulfur dioxide under Venus atmospheric conditions will be used to create a new model that accurately characterizes the centimeter- and millimeter-wavelength properties of sulfur dioxide. After the completion of the millimeter-wavelength laboratory measurements, both centimeter- and millimeter-wavelength data will be put into an optimization algorithm to create the best model estimate of the absorptivity of sulfur dioxide under Venus conditions. 

\subsection{Facilities Required}
All of the facilities required for this work currently exist in the Planetary Atmospheres Laboratory at The Georgia Institute of Technology. The facilities include millimeter-wavelength test equipment, a Fabry-Perot resonator, a temperature chamber, pressure and temperature gauges, and a computer running Matlab. Additional resources that are required and available are gas cylinders of carbon dioxide and sulfur dioxide. Computing resources necessary for model development are available at the Planetary Atmospheres Laboratory.

\subsection{Milestone}

\subsubsection*{Completed Work}
\begin{itemize}
\item Completed setup of millimeter-wavelength system: January 2014
\item Completed 3-4 millimeter-wavelength opacity of SO$_2$ under simulated Venus conditions: February 2014
\item Completed 2-3 millimeter-wavelength opacity of SO$_2$ under simulated Venus conditions: April 2014
\item Diagnosed issue with SO$_2$ pressure regulator and modified data accordingly: April 2014
\item Submit thesis proposal: April 2014
\end{itemize}
\enlargethispage{\baselineskip}
\subsubsection*{Remaining Work}
\begin{itemize}
\item Verify constituent inventory in SO$_2$ bottle and modify data accordingly: April 2014
\item Develop consistent model for opacity of SO$_2$ incorporating the millimeter-wavelength measurements developed as part of this work and the centimeter-wavelength measurements made previously: June 2014
\item Submit journal paper detailing the millimeter-wavelength measurements and the opacity model: November 2014
\item Submit Master's Thesis: December 2014
\end{itemize}