% !TEX TS-program = pdflatex
% !TEX encoding = UTF-8 Unicode

% This is a simple template for a LaTeX document using the "article" class.
% See "book", "report", "letter" for other types of document.

\documentclass[11pt]{article}

\usepackage[utf8]{inputenc} % set input encoding (not needed with XeLaTeX)

%%% Examples of Article customizations
% These packages are optional, depending whether you want the features they provide.
% See the LaTeX Companion or other references for full information.

%%% PAGE DIMENSIONS
\usepackage{geometry} % to change the page dimensions
\geometry{letterpaper} % or letterpaper (US) or a5paper or....

\usepackage[]{graphicx} % support the \includegraphics command and options

% \usepackage[parfill]{parskip} % Activate to begin paragraphs with an empty line rather than an indent

%%% PACKAGES
\usepackage{booktabs} % for much better looking tables
\usepackage{array} % for better arrays (eg matrices) in maths
\usepackage{paralist} % very flexible & customisable lists (eg. enumerate/itemize, etc.)
\usepackage{verbatim} % adds environment for commenting out blocks of text & for better verbatim
\usepackage{subfig} % make it possible to include more than one captioned figure/table in a single float
% These packages are all incorporated in the memoir class to one degree or another...
\usepackage{mathtools}
\usepackage{listings}
\usepackage{amsmath}
\usepackage{setspace}
\doublespacing
% or:
%\onehalfspacing
%\usepackage[authoryear]{natbib}
\usepackage{hyperref}
\usepackage{float}
\usepackage{epsfig}
\usepackage{import}



%%% HEADERS & FOOTERS
\usepackage{fancyhdr} % This should be set AFTER setting up the page geometry
\pagestyle{fancy} % options: empty , plain , fancy
\renewcommand{\headrulewidth}{0pt} % customise the layout...
\lhead{}\chead{}\rhead{}
\lfoot{}\cfoot{\thepage}\rfoot{}

%%% SECTION TITLE APPEARANCE
\usepackage{sectsty}
\allsectionsfont{\sffamily\mdseries\upshape} % (See the fntguide.pdf for font help)
% (This matches ConTeXt defaults)

%%% ToC (table of contents) APPEARANCE
\usepackage[nottoc,notlof,notlot]{tocbibind} % Put the bibliography in the ToC
\usepackage[titles,subfigure]{tocloft} % Alter the style of the Table of Contents
\renewcommand{\cftsecfont}{\rmfamily\mdseries\upshape}
\renewcommand{\cftsecpagefont}{\rmfamily\mdseries\upshape} % No bold!
\usepackage{datetime}

\newdateformat{mydate}{\monthname[\THEMONTH] \THEYEAR}


\title{THE MILLIMETER-WAVELENGTH SULFUR DIOXIDE ABSORPTION SPECTRA MEASURED UNDER SIMULATED VENUS CONDITIONS }
\author{A Masters Thesis Proposal\\
Presented to\\
The Academic Faculty \\
by \\
Amadeo Bellotti}
\date{\mydate\today} % Activate to display a given date or no date (if empty),
         % otherwise the current date is printed 

\begin{document}
\maketitle

%\begin{abstract}
%The objective of the proposed research is to develop a mathematical model that accurately estimates the opacity of sulfur dioxide in a carbon dioxide atmosphere under conditions characteristic of the Venus troposphere based on extensive laboratory measurements. High precision measurements of the millimeter-wavelength properties of sulfur dioxide are being conducted under multiple pressure and temperatures. These measurements are being conducted in both W-band and F-band (2-3 and 3-4 millimeter-wavelengths). The results of this research will significantly improve the understanding of the millimeter-wavelength emission spectrum of Venus and possibly determine the source of variations in the Venus millimeter-wavelength emissions. 
%\end{abstract}
\newpage
\tableofcontents
\chapter{Introduction}

%\section{Introduction}
Active and passive microwave remote sensing techniques have been extensively used in the study of our sister planet, Venus. Unlike Earth's atmosphere, the Venus atmosphere is mostly comprised of gaseous carbon dioxide (CO$_2$). CO$_2$ comprises 96.5\% of the atmosphere along with gaseous nitrogen (N$_2$) at about 3.5\%. The Venus atmosphere has multiple trace constituents such as sulfur dioxide (SO$_2$), carbon monoxide (CO), water vapor (H$_2$O), carbonyl sulfide (OCS), and sufuric acid vapor (H$_2$SO$_4$) \cite{Suleiman-thesis}.

Two sulfur-bearing compounds dominate the millimeter-wave emission from Venus: sulfur dioxide (SO$_2$) and gaseous sulfuric acid (H$_2$SO$_4$). At higher pressures H$_2$SO$_4$ thermally dissociates, forming H$_2$O and SO$_2$, both of which exhibit relatively small amounts of microwave absorption at the abundance levels present in the Venus atmosphere. Thus, in the deep atmosphere, only SO$_2$ and CO$_2$ have the potential to affect the observed microwave emission.

Utilizing the millimeter-wavelength system at the Planetary Atmospheres Laboratory at Georgia Institute of Technology it is possible to simulate the upper troposphere of Venus and take precision measurements of the millimeter-wavelength properties of sulfur dioxide. Using the measurements, a model can be created that accurately predicts the opacity of sulfur dioxide in the Venus atmosphere. This model will make it possible to determine the source of variations in the Venus millimeter-wavelength emission, such as were observed by Sagawa \cite{observations}.

\section{Background and Motivation}

\section{Organization}
\chapter{Theoretical Basis and Previous Work}

\section{Physical Structure of SO$_2$}
\section{Physical Structure of CO$_2$}
\section{Van Vlevk and Weisskopf Model}
\chapter{Experiment Design, Theory, and Results}

Verifying millimeter-wavelength absorption spectrum of SO$_2$ is important for the study of the atmosphere of Venus. Making measurements under simulated Venus conditions assures the accuracy of any model derived from such measurements.
 Described below is the theory, laboratory equipment, measurement procedure and derived uncertainties in the measurements of the millimeter-wavelength absorptivity of gaseous sulfur dioxide under simulated Venus conditions.

\section{Measurement Theory}

In this experimental program, quality factor (Q) of a resonant mode of a resonator is used to measure the absorption of a gas or gas mixture \cite{high-sensitivity}. The quality factor of a resonance is given by \cite{matching}

\begin{equation} \label{eq:qlong}
Q = \frac{2\pi f_0 \textnormal{ x Energy Stored}}{\textnormal{Average Power Loss}}
\end{equation}

\noindent where $f_0$ is the resonant frequency. The Q of a resonance can be measured directly from $f_0$ by dividing it by its half-power bandwidth (HPBW).

\begin{equation} \label{eq:qshort}
Q = \frac{f_0}{HPBW}
\end{equation}

\noindent The Q of a lossy gas ($\epsilon'/\epsilon''$) and its opacity are related by
\begin{equation} \label{eq:alphaapprox}
\alpha \approx \frac{\epsilon'' \pi}{\epsilon' \lambda} = \frac{1}{Q_{gas}} \frac{\pi}{\lambda}
\end{equation}

\noindent where $\epsilon'$ and $\epsilon''$ are the real and imaginary permittivity of the gas, $\lambda$ is the wavelength in km, and $\alpha$ is the absoptivity of the gas in Nepers/km (1 Neper = 8.686 dB). Since Q can be affected by more than just the gas added, the Q of the gas-filled resonator is given by

\begin{equation} \label{eq:qloaded}
\frac{1}{Q_{loaded}^m} = \frac{1}{Q_{gas}} + \frac{1}{Q_{r}} + \frac{1}{Q_{ext1}} +\frac{1}{Q_{ext2}}
\end{equation}

\noindent where $Q_{loaded}^m$ is the measured quality factor of a resonance in the presence of a test gas, $Q_{gas}$ is the quality factor of the gas under test, $Q_{r}$ is the quality factor of the resonator in the absense of coupling losses, and $Q_{ext1}$ and $Q_{ext2}$ are the external coupling losses. Since the resonator used is symmetric, it is safe to assume $Q_{ext1} = Q_{ext2}$. Coupling losses can be derived from the transmissivity $t = 10^{-S/10}$, where $S$ is the measured insertion loss of the resonator in decibels (dB) at the frequency of a particular resonance using the following relationship

\begin{equation} \label{eq:t}
t = \left[ w \frac{Q^m}{Q_{ext}} \right]^2,
\end{equation}

\begin{equation} \label{eq:qext}
Q_{ext} = \frac{2Q^m}{\sqrt{t}}
\end{equation}

\noindent $Q_r$ is related to the measured Q at a vacuum by

\begin{equation}\label{eq:qvac}
\frac{1}{Q_{vac}^m} =  \frac{1}{Q_{r}} + \frac{1}{Q_{ext1}} +\frac{1}{Q_{ext2}}
\end{equation}

\noindent where $Q_{vac}^m$ is the measured Q at a vacuum. Substituting equation \ref{eq:qext} into equations \ref{eq:qloaded} and \ref{eq:qvac} gives

\begin{equation}\label{eq:qgas}
\frac{1}{Q_{gas}} = \frac{1 - \sqrt{t_{loaded}}}{Q^m_{loaded}} - \frac{1-\sqrt{t_{vac}}}{Q_{vac}^m}
\end{equation}

\noindent where $t_{loaded}$ and $t_{vac}$ are the transmissivity of the resonance taken in loaded and vacuum conditions respectively. When gas is added to the resonator there is a shift in the center frequency corresponding to the refractive index of the test gas. Since the quality factor is reliant on the center frequency this will affect the comparison between the two measurements. This effect is called dielectric loading \cite{h2s-labmesurements}. Dielectric matching can be achieved by performing additional measurements of the quality factor with a lossless gas present. Adding the lossless gas shifts the center frequency of the resonances, and by adding more or less gas the center frequency can be adjusted to be exactly the same as the lossy gas. These measurements are used in place of the vacuum measurements in equation \ref{eq:qgas} and by converting Nepers/km to dB/km we can rewrite equation \ref{eq:alphaapprox} as

\begin{equation} \label{eq:alphamatch}
\alpha = 8.686 \frac{\pi}{\lambda}\left(\frac{1 - \sqrt{t_{loaded)}}}{Q^m_{loaded}} - \frac{1-\sqrt{t_{matched}}}{Q_{matched}^m} \right) dB/km
\end{equation}

\section{Millimeter-Wavelength Measurement System}

The high-sensitivity millimeter-wavelength system used for measuring the opacity of gaseous sulfur dioxide under Venus conditions is similar to the one used by Devaraj and Steffes \cite{system-description} \cite{Devaraj-thesis}. The system is comprised of two subsystems for measuring different bands of the millimeter-wavelength spectrum (W-band/F-band). The simulator consists of a glass pressure chamber capable of withstanding up to 3 bars of pressure along with a temperature chamber capable of operating up to 400K. The W-band subsystem is used for measurements in the 3-4 millimeter-wavelength range while the F-band system is used for the 2-3 milimeter-wavelength range. The following sections describe each subsystem and their components. 

\subsection{W-band Subsystem}

The W-band measurement system is used to measure the 3-4 mm-wavelength properties of sulfur dioxide is shown in figure \ref{fig:wbandimage}.

A synthesized swept signal generator (HP 83650B) is used to generate a signal in the 12.5-18.3 GHz range which is fed through a times-six active multiplier chain (AMC) via low-loss, high frequency coaxial cables. The radio frequency (RF) signal from the output port of the Fabry-Perot resonator (FPR) is fed to a QuinStar Technology QMH series harmonic mixer. The local oscillator (LO) and the intermediate frequency (IF) are connected via an external diplexer. The harmonic mixer is locked to the 18th harmonic of the spectrum analyzer LO and is used in the ``external mixer'' mode with the spectrum analyzer (HP 8564E). 

\begin{figure}[H]
    \centering
	\includegraphics[width=0.7\textwidth]{./images/w-bandsystem.png}
	\caption{Block diagram of the W band measurement system. Solid lines represent the electrical connections and the arrows show the direction of the signal propagation. Valves controlling the flow of gasses are shown by small crossed circles.}
    \label{fig:wbandimage}
\end{figure}


\subsection{F-band Subsystem}
The F-band measurement system is used to measure the 2-3 mm-wavelength properties of sulfur dioxide and is shown in figure \ref{fig:fbandimage}.

The swept signal generator (HP 83650B) is used to generate a signal in the 33-50 GHz range which
is amplified and fed through a frequency tripler. The output of the tripler is fed to the input end of the FPR. The RF signal from the output port of the FPR is fed to a harmonic mixer which can operate with an LO frequency as high as 18 GHz. An external diplexer is used to combine the IF and LO signals. For a particular RF and IF frequency,  the LO frequency can be computed using

\begin{equation} \label{eq:fbandlo}
f_{LO} = \frac{f_{RF} - f_{IF}}{N_H	}
\end{equation}

\noindent where N$_H$ is the lowest integer such that $f_{lo} < 18 GHz$.

\begin{figure}[H]
    \centering
	\includegraphics[width=0.7\textwidth]{./images/f-bandsystem.png}
	\caption{Block diagram of the F band measurement system. Solid lines represent the electrical connections and the arrows show the direction of the signal propagation. Valves controlling the flow of gasses are shown by small crossed circles.  }
    \label{fig:fbandimage}
\end{figure}

\section{Data Handling Subsystem}

The data acquisition system consists of a computer connected to the spectrum analyzer (HP 8564E), swept signal generator (HP 83650B), and continuous wave (CW) signal generator (HP 83712B, the local oscillator for the F-Band system) via a general purpose interface bus (GPIB). The instruments are controlled via Matlab script and their appropriate programming language. The software used is similar to Devaraj and Steffes \cite{system-description} \cite{Devaraj-thesis} with modifications for equipment changes.

\section{Measurement Procedure}

The most important prerequisite for performing measurement of gas properties is ensuring a leak-proof system. This is done through two methods, the first is by drawing a vacuum inside the FPR and verifying the integrity of the vacuum over time. The second way is by adding a positive pressure of CO$_2$ to the system and making sure there are no leaks in any of the connectors and valves. Ensuring a leak-proof system allows for not only precise measurements but also ensures no toxic gases are released into the testing environment.

After the system is ensured to be leak-proof and at a stable temperature, a vacuum is drawn and a measurement is taken using the appropriate subsystem (W-band for 3-4 mm-wavelengths, F-band for 2-3 mm-wavelengths). This allows for a base line measurement of the FPR's resonances and the Quality factor. Once this baseline is established the gas under test is added to the system.

Once the gas temperature has stabilized, another set of tests measuring the resonant frequencies along with the quality factors is taken. More gas is added and the procedure is repeated until all suitable pressures are taken. A vacuum is drawn once again but this time it is pumped overnight due to the gas being tested (SO$_2$) and its properties of ``sticking'' (or adsorbing) to metal. Another vacuum measurement is taken to measure any possible system error.

Once the second vacuum measurements are taken CO$_2$ is then added to the chamber until the resonances are matched to the same frequency of our test gas. Again measurements are taken and this is repeated for every pressure of the test gas. Once this is finished a vacuum is again drawn and another test is taken. 

Lastly the system is set up for a transmissivity test where we measure t (equation \ref{eq:t}) for each given resonant frequency. The system is then set back up and is ready for a new test. Reference table \ref{tab:testmatrix} for the testing matrix being used.


%\section{Preliminary Results}

Currently the millimeter-wavelength system is completely operational in the Planetary Atmospheres Laboratory at The Georgia Institute of Technology. Using this system, high precision measurements of SO$_2$'s millimeter-wavelength absorption have recently been completed as part of this work. A preliminary model of SO$_2$'s absorption properties is available from Suleiman's previous work on microwave laboratory measurements \cite{Suleiman-thesis}. Additionally measurements  of SO$_2$'s centimeter-wavelength absorption have recently been taken under deep Venus conditions \cite{so2-cent-lab} \cite{so2-cent-model}. 

\subsection{Millimeter-Wavelength Results}

Only one previous measurement of SO$_2$'s mm-wavelength opacity under Venus simulated conditions has been done (see Fahd et. al. 1991) \cite{fahd-so2} This measurement was done using only one frequency (94.1 GHz) in the mm-wavelength spectra.

In our experiment, eight different frequencies have been already tested using the millimeter-wavelength system measuring 100 mbar of SO$_2$ along with separate tests at 1 bar CO$_2$ combined with the SO$_2$  and 2 bar of CO$_2$ combined with the SO$_2$. This allows for a comparison with Fahd's model and with Suleiman's model at higher frequencies.  

\subsubsection{Absorption Model}

The goal of the laboratory measurements is to create a mathematical model that accurately estimates the opacity of sulfur dioxide in a carbon dioxide atmosphere under all possible conditions of temperature, pressure, concentration, and frequency (fTPC space). For the data fitting process we will use data taken with the FPR along with data from the Planetary Atmospheres Laboratory centimeter-wavelength system  \cite{so2-cent-lab} \cite{so2-cent-model} to create a model that best fits the fTPC space. 

Extrapolating models for SO$_2$'s absorption into the mm-wavelength allows for a good starting point in the model creation process. As visible in the following figures, the absorption model matches the same shape as previous models but is lower by about 20\%. Finding a unified model for SO$_2$'s absorption will compensate for this change.

Figures \ref{fig:so2-116}, \ref{fig:so2-943}, and \ref{fig:so2-1987} show the initial data taken for SO$_2$ opacity in the 3-4 mm-wavelength range. It is clear that previous absorption models work well in predicting the shape of the millimeter-wavelength absorption spectrum of SO$_2$.

\begin{figure}[H]
    \centering
	\includegraphics[width=0.8\textwidth]{./plots/35C_W_High/{0.116-modelComparison}.eps}
	\caption{Measured absorption spectrum given 116 mbar of SO$_2$ at 308K in the W-band range. Shown for comparison are models from Devaraj (2011), Suleiman et al (1996), and Fahd and Steffes (1992).}
    \label{fig:so2-116}
\end{figure}

\begin{figure}[H]
    \centering
	\includegraphics[width=0.8\textwidth]{./plots/35C_W_High/{0.943-modelComparison}.eps}
	\caption{Measured absorption spectrum given 116 mbar of SO$_2$ and 827 mbar of CO$_2$ at 308K in the W-band range.
	Shown for comparison are models from Devaraj (2011), Suleiman et al (1996), and Fahd and Steffes (1992).}
    \label{fig:so2-943}
\end{figure}

\begin{figure}[H]
    \centering
	\includegraphics[width=0.8\textwidth]{./plots/35C_W_High/{1.987-modelComparison}.eps}
	\caption{Measured absorption spectrum given 116 mbar of SO$_2$ and 1871 mbar of CO$_2$ at 308K in the W-band range.
	Shown for comparison are models from Devaraj (2011), Suleiman et al (1996), and Fahd and Steffes (1992).}
    \label{fig:so2-1987}
\end{figure}
\chapter{Model Fitting and Modifications}
In total, 36 data sets were taken at 2-4 mm-wavelength and at two temperatures (12 at $\sim$308 K and 24 at $\sim$343 K) . This, along with data taken at the centimeter-wavelength by Steffes et al. \cite{Steffes-2015} (10 data sets at $\sim$435 K, 10 data sets at $\sim$490 K, and 5 data sets at $\sim$550 K), were used in finding the best-fit model.

Before creating a new formalism for the absorption of SO$_2$ in a CO$_2$ atmosphere, analysis of previous models was conducted. The Van Vleck and Weisskopf Model (VVW) used by Fahd and Steffes \cite{Fahd-1991} with the new JPL rotational line catalog (Pickett, et al. \cite{Pickett-1998}) was found to fit 85.88\% of all 500 data points within $2\sigma$ uncertainty (95\% confidence). Consideration of the model analysis process and the final model are presented.
\section{Measurement Uncertainties}

There are five uncertainties for absorptivity measurements using the  centimeter and millimeter wavelength systems (Hanley \cite{Hanley-thesis}) at the Planetary Atmospheres Laboratory at The Georgia Institute of Technology: instrumentation errors and electrical noise ($Err_{inst}$), errors in dielectric matching ($Err_{diel}$), errors in transmissivity measurement ($Err_{trans}$), errors due to resonance asymmetry ($Err_{asym}$), and errors in measurement conditions ($Err_{cond}$) resulting from uncertainties in temperature, pressure, and mixing ratio. The term $Err$ is used for representing $2\sigma$ uncertainties.

Instrumental errors and electrical noise are due to the limited sensitivity of the electrical devices and their ability to accurately measure bandwidth ($BW_{measured}$) and the center frequency ($f_o$). Electrical noise arises from the limited-stability frequency references and the noise of the internal electronics. Electrical noise is uncorrelated and the best estimate of instrumental uncertainty is the variance of multiple measurements. The variance of the error estimate is given by the sample variance ($S^2_N$) weighted by the confidence coefficient ($B$) as
\begin{equation}\label{eq:sigman}
\sigma^2_N = B \frac{S^2_N}{N_{samples}}
\end{equation}

\noindent where $N_{samples}$ is the number of independent measurements of the sample. For the millimeter-wavelength system, five sets of independent measurements of each resonance are taken. A confidence coefficient ($B$) of 2.776 is used. This corresponds to the 95\% confidence interval ($2\sigma$). The center frequency standard deviation is very small and its effect on the uncertainty in $Q$ is negligible. Therefore, $S_N$ is the sample standard deviation of the bandwidth of the measurements.

The HP 8564E spectrum analyzer is used for measuring the resonances in the millmeter-wavelength system. It's manufacturer-specified instrumental uncertainties are the $3\sigma$ values \cite{Hewlett-Packard}. The $3\sigma$ standard deviation for the center frequency and bandwidth are estimated by 

\begin{equation}\label{eq:sigmao}
Err_o \leq \pm (f_o \times f_{ref\:acc} + 0.05 \times SPAN + 0.15 \times RBW +10 ) (Hz)
\end{equation}
\begin{equation}\label{eq:sigmabw}
Err_{BW} \leq \pm (BW_{measured} \times f_{ref\;acc} + 4 \times N_H +2 \times LSD ) (Hz)
\end{equation}

\noindent where $f_{ref\;acc}$ is given as
\begin{equation}\label{eq:frefacclong}
\begin{split}
f_{ref\;acc} = (aging \times {time\;since\;calibration}) + {inital\;achievable\;accuracy} \\
+ {\;temperature\;stability}
\end{split}
\end{equation}

\noindent and $f_o$, SPAN, RBW, $N_H$, and LSD are the center frequency, frequency span, resolution bandwidth, harmonic number, and least significant digit of the bandwidth measurement, respectively. LSD is calculated as 
\begin{equation}
LSD = 10^x
\end{equation} 
where x is the the smallest positive integer value of x such that SPAN $< 10^{x+4}$. For SPAN $\leq 2$ MHz$\times N_h$, Equation \ref{eq:sigmao} becomes 
\begin{equation}\label{eq:sigmaosmall}
Err_o \leq \pm (f_o \times f_{ref\:acc} + 0.01 \times SPAN + 0.15 \times RBW +10 ) (Hz)
\end{equation}
For the spectrum analyzer used, $f_{ref\;acc}$ reduces to

\begin{equation}\label{eq:frefacc}
f_{ref\;acc} = (10^{-7} \times {years\;since\;calibrated}) + 3.2\times 10^{-8}
\end{equation}

The worst case scenario is used to transform the uncertainty in center frequency and bandwidth for both loaded and dielectrically matched measurements into an uncertainty in absorptivity as described in DeBoer and Steffes \cite{DeBoer-Steffes}.

\begin{equation}
Err^2_\Psi = \langle {F_l^2}\rangle + \langle {F_m^2}\rangle -\langle {F_l F_m}\rangle
\end{equation}

\noindent where
\begin{equation}
\langle {F_i^2}\rangle = \frac{\Upsilon_i^2}{f_{oi}^2}
\left[ \frac{Err_o^2}{Q_l^2} + Err_{BW}^2 + Err_{Ni}^2 + \frac{2Err_o Err_{BW}}{Q_i} \right], i= l,m
\end{equation}
\begin{equation}
\langle {F_l F_m}\rangle = -\frac{\Upsilon_l \Upsilon_m}{f_{ol} f_{om}}
\left[ \frac{Err_o^2}{Q_i Q_m} + Err_{BW}^2 + \frac{Err_o Err_{BW}}{Q_l} + \frac{Err_o Err_{BW}}{Q_m}\right]
\end{equation}
\begin{equation}
Q_i = \frac{f_{oi}}{f_{BWi}}, i = l,m
\end{equation}
\begin{equation}
\Upsilon_i = 1- \sqrt{t}, i = l,m
\end{equation}
where $l,m$ denote loaded and dielectrically matched cases respectively and $f_{ol,om}$ and $f_{BWl,BWm}$ represent center frequency and bandwidth of loaded and dielectrically matched cases respectively. The $2\sigma$ uncertainty of the measured gas absorption due to instrumental errors and electrical noise is given by
\begin{equation}
Err_{inst} = \pm \frac{8.686\pi}{\lambda}Err_\Psi\;(dB/km)
\end{equation}
where $\lambda$ is the wavelength in km. 

Errors in dielectric matching occur when the when the center frequency of the matched measurements are not precisely aligned with the center frequency of the loaded measurement. Since the Q of the resonator can vary slightly, this causes an uncertainty in the Q of the matched measurement at the true center frequency of the loaded measurement. The method used to calculate the magnitude of this effect is similar to Hanley \cite{Hanley-thesis}. While this error is the smallest due to the high precision of the software controlled matching, it is important to calculate and account for. The magnitude of this effect is calculated by comparing the Q of the three vacuum measurements to that of the dielectric matched measurements

\begin{equation}
\left(\frac{dQ}{df} \right)_i = \left|\frac{Q_{vac,i} - Q_{matched,i}}{f_{vac,i} - f_{matched,i}} \right| \textnormal{ for } i = 1,2,3
\end{equation}

The maximum of the three values is used to calculate a $dQ$ value

\begin{equation}
dQ = \left(\frac{dQ}{df} \right)_{max} \times |f_{loaded} - f_{matched}|
\end{equation}
where $f_{loaded}$ and $ f_{matched}$ are the center frequencies of the resonances under loaded and matched conditions. The error in absorptivity due to imperfect dielectric matching is then computed by propagating $\pm dQ$ through Equation \ref{eq:alphamatch}.
% \times \left| \left( \frac{1-\sqrt{t_{loaded}}}{Q^m_{loaded}} - \frac{1-\sqrt{t_{matched}}}{Q^m_{matched} + dQ} \right) \left|
\begin{equation}
\begin{split}
Err_{diel} &= \frac{8.686 \pi}{\lambda} 
\\ &\times \left| \left( \frac{1-\sqrt{t_{loaded}}}{Q^m_{loaded}} - \frac{1-\sqrt{t_{matched}}}{Q^m_{matched} + dQ} \right) - \left( \frac{1-\sqrt{t_{loaded}}}{Q^m_{loaded}} - \frac{1-\sqrt{t_{matched}}}{Q^m_{matched} - dQ} \right) \right|\\
 &(dB/km)
\end{split}
\end{equation}

Transmissivity errors are due to the uncertainties in the measurement amplitude. This is caused by variations in gains of losses of the millimeter-wavelength instruments (signal generators and spectrum analyzer), cables, adapters, and waveguides used in this system. This is done by taking multiple test measurements of signal loss through the system without the FPR and finding the standard deviation ($S_N$) of the signal loss and weighing it by its confidence coefficient
\begin{equation}
Err_{msl} = \frac{4.303}{\sqrt{3}}S_N
\end{equation}

For the millimeter-wavelength system, the signal level measurements involve sampling the RF power with a WR-10 20 dB directional coupler to feed the harmonic mixer for down-conversion and detection. While this ensures that the input to the harmonic mixer does not exceed its maximum allowed input power of -10 dBm, the WR-10 20dB directional coupler does not uniformly sample the input signal throughout the entire frequency range. To compensate for this, an aditional 1.5 dB uncertainty is added to insertion loss error. The signal generator has a temperature stability of 1 dB/$10^\circ$ C, but an internal temperature equilibrium is reached after two hours \cite{Hewlett-Packard}. Since the measurements units are stored at a constant temperature this uncertainty can be disregarded. The total uncertainty in insertion loss for the millimeter-wavelength system is calculated by
\begin{equation}
Err_{ins\;loss} = Err_{msl} +1.5\;(dB)
\end{equation}

The error in insertion loss is used to compute the transissivity error
\begin{equation}
Err_{t,i} = \frac{1}{2} ( 10^{-S_i - Err_{ins\;loss}} - 10^{-S_i + Err_{ins\;loss}}) , i=l,m
\end{equation}
where l, m are the loaded and matched cases, respectively, and S is the insertion loss of the resonator. This is used to compute the $2\sigma$ uncertainties in opacity and is expressed as
\begin{equation}
\begin{split}
Err_{trans} &= \frac{8.686 \pi}{2\lambda}\\ 
 &\times \left| \left( \frac{\sqrt{t_l + Err_{t,l}}- \sqrt{t_l - Err_{t,l}}}{Q^m_{loaded}} - \frac{\sqrt{t_m - Err_{t,m}}- \sqrt{t_m + Err_{t,m}}}{Q^m_{matched}} \right) \right|\\
 &(dB/km).
\end{split}
\end{equation}

Errors from asymmetry are due to the asymmetric nature of the resonances. These are more prominent at low temperatures and short wavelengths. Errors due to the asymmetry result from the disproportionate asymmetric broadening of the loaded measurements compared to the matched measurements. Equivalent full bandwidths based on assuming symmetry of the high and low sides of the resonances are calculated as
\begin{equation}
BW_{high} = 2 \times (f_{high} - f_{center})
\end{equation}
\begin{equation}
BW_{low} = 2 \times (f_{center} - f_{low})
\end{equation}
where $BW_{high}, BW_{low}, f_{high}, f_{center}$, and $f_{low}$ are the high bandwidth, low bandwidth, higher frequency half power point, center frequency, and lower frequency half power point, respectively. For a perfectly symmetric resonance, $BW_{high} = BW_{low}$. The difference between the opacities calculated using $BW_{high}$ and $BW_{low}$ is defined as $Err_{asym}$ and is calculated by
\begin{equation}
\begin{split}
Err_{asym} &= \frac{8.686 \pi}{\lambda} 
\\ &\times \left| \left( \frac{1-\sqrt{t_{loaded}}}{Q^m_{loaded,high}} - \frac{1-\sqrt{t_{matched}}}{Q^m_{matched,high}} \right) - \left( \frac{1-\sqrt{t_{loaded}}}{Q^m_{loaded,low}} - \frac{1-\sqrt{t_{matched}}}{Q^m_{matched,low}} \right) \right|\\
 &(dB/km)
\end{split}
\end{equation}
Where $Q^m_{matched,high/low}$ and $Q^m_{loaded,high/low}$ are the measured Q's evaluated using the high and low bandwidths for loaded and matched cases. 

The uncertainties in measured temperature, pressure, and concentration in the millimeter-wavelength system contribute to the total uncertainty due to the measurement conditions ($Err_{cond}$). While uncertainties in measurement conditions do not directly affect the measurements of millimeter-wavelength absorptivity, they still need to be accounted for when evaluating the opacity formalisms. It is computed by
\begin{equation}
Err_{cond} = \sqrt{Err_{temp}^2 + Err_{p}^2 + Err_{c}^2 } (dB/km)
\end{equation}
with $Err_{temp}$, $Err_{p} $, and $ Err_{c}$ representing the 2$\sigma$ uncertainties  in temperature, pressure, and concentration (or mole fraction) respectively. Each of these are calculated by taking the maximum modeled opacity with each uncertainty minus the minimum modeled opacity and halving the difference. 

Temperature was measured using a T type thermocoupler along with a Wavetek 23XT voltmeter. The voltmeter has a temperature accuracy of $\pm (1\% + 2^\circ C)$. Since the voltmeter has a cold compensation circuitry it is unnecessary to correct for ambient temperature. The temperature inside the test vessel is stable enough that it does not drift a significant amount during the hour it takes to run a test. The uncertainty in temperature is calculated by
\begin{equation}
T = T_{read} \pm ( T_{read} \times 1\% + 2)
\end{equation}
Where $T_{read}$ is the temperature (in $^\circ$C) displayed by the Wavetek 23XT voltmeter.

Pressure was measured using an Omega DPG-7000 which has an accuracy of $\pm 0.05\%$FS (full scale). Since this pressure gauge measures pressure relative to ambient it is necessary to take a measurement before and after each test to ensure that the ambient pressure did not change significantly during the test. 
The average change in pressure during a test was at most 2 mbar. The cause of this change was identified as a change in ambient pressure during the test. Since the Omega DPG-7000 is a relative pressure gauge it was necessary to track ambient pressure. 
A vacuum was ensured by comparing the Omega DPG-7000 reading to that of an absolute pressure gauge (Druck DPI 104). The Druck has an accuracy of $\pm 0.05\%$FS as well as a resolution of $\pm 1$ mbar. The uncertainty in pressure reading is calculated by
\begin{equation}
P = P_{read} \pm ( P_{FS} \times .05\% + 3)
\end{equation}
Where $P_{FS}$ is the full scale pressure of the Omega DPG-7000 (3.08 bars).

Since $Err_{cond}$ is dependent on the opacity model, this uncertainty is maintained separately from $Err_{tot}$. Thus the total 95\% confidence for the measurement uncertainty is expressed in dB/km as per Hanley \cite{Hanley-thesis}
\begin{equation}
Err_{tot} = \sqrt{Err_n^2 + Err_{diel}^2 + Err_{trans}^2 + Err_{asym}^2} \;(dB/km).
\end{equation}


\section{Model Analysis Process}

The models considered in this comparison are the Van Vleck-Weisshopf model (using coefficients from Fahd and Steffes \cite{Fahd-1991}) and the Ben-Reuven model as calculated by Suleiman et al. \cite{Suleiman-1996}. The comparison of these models are done using a L$_2$ norm analysis. 

The following compliance function was used to calculate the number of data points that each model encompassed,
\begin{equation}
\textbf{1}_{model}(\alpha) = \left\{
     \begin{array}{lr}
       1 & : |\alpha_{measured} - \alpha_{model}| \leq \sqrt{Err_{tot}^2 + Err_{cond}^2 }\\
       0 & : |\alpha_{measured} - \alpha_{model}| > \sqrt{Err_{tot}^2 + Err_{cond}^2 }
     \end{array}
   \right.
\end{equation}
where $\textbf{1}_{model}$ is the compliance function for each model, and $\alpha_{measured}$, $\alpha_{model}$ is the measured absorption and the calculated absorption, respectively. $Err_{tot}$ and $Err_{cond}$ are the systematic and conditional errors as described previously. The percentage of data points that each model encompasses can be calculated using,
\begin{equation}
Per_{model} = \frac{\sum_{i=1}^N \textbf{1}_{model}(\alpha_i)}{N}\times 100\%
\end{equation}
where $Per_{model}$ is the percentage of data points that the model fits and $N$ is the total number of data points. The final results are summarized in Table \ref{tab:model-comp}.

\begin{sidewaystable}[p]

\caption{The percentage of the measured data points within $2\sigma$ uncertainty of the different models}
  \begin{tabular}{l | c c | r}
  \hline
  \hline
  SO$_2$ opacity model & Centimeter-Wavelength (1-8 GHz) & Millimeter-Wavelength (80-150 GHz) &Total\\
  \hline
   Fahd and Steffes (1992)	 & 82.95\%	& 	88.89\% & 84.49\%\\%85.88\%\\
  Suleiman (1997)			 & 62.98\%	&	88.10\% & 70.37\%\\
  \hline
  \hline
  \end{tabular}
  \label{tab:model-comp}
\end{sidewaystable}

 
\clearpage
\section{Experimental Results}
High accuracy laboratory measurements of the temperature and pressure dependence of the millimeter-wavelength absorption of gaseous SO$_2$ in a CO$_2$ atmospheres have been conducted at 308 and 345 K at pressures from 30 mbar to 3 bars for wavelengths between 2-4 milimeters. The following plot show the results of these absorptivity measurements with the two sigma uncertainties. For comparison purposes these tables also show two known formalisms of SO$_2$'s absorptivity. One developed by Suleiman et al. 1997 \cite{Suleiman-thesis} and the second by Fahd and Steffes \cite{Fahd-thesis}. 

\subsection{Accuracy of Constituents}
Originally when the data was taken, the results fit the lineshapes of these two model's but the absorption calculated was much lower then predicted. After careful considerations of what could have caused this the SO$_2$ bottle was shipped back to Airgas\texttrademark for analysis. It was concluded that the SO$_2$ bottle used was comprised of 84.7\% SO$_2$ and 15.3\% N$_2$. Luckily N$_2$ has little to no absorptivity in the millimeter-wavelength domain so it is safely ignored.

When this factor was then taken into account it became possible to compute both model's with the correct SO$_2$ mixing ratio. When this was done the data fit the model's lineshape and absorption. This is also the reason that the SO$_2$ mixing ratio and CO$_2$ mixing ratio do not add up to 100\%.

The CO$_2$ tank used was the same tank used in Steffes et al. 2014 \cite{Steffes-2014}. Since the CO$_2$ absorption measured in the paper matched the formalism and the refractive index of the gas matched that of CO$_2$ it is assumed that the tank is pure CO$_2$.

\subsection{Cleaning of Data}

The systems had a tenancy to lose resonate peaks when the temperature changed. This causes the variable data points in different plots. Data was also removed if it was not physically possible (negative or huge, $>1000$dB/km, opacity) and if the error bars were too large (encompassing 50 orders of magnitude). These were caused by the system either not finding the expected resonant peaks and/or (in the case of the F-Band system) interference due to outside sources. While both of these issues were minimized as much as possible they still occurred. 

\begin{figure}[p]
 \centering 
\includegraphics[width=0.7\textwidth]{./model/results/{{308.75K-0.03bar-25so2-modelComparison}}} 
 \caption{Opacity data  using the 2-3 mm-wavelength system for a mixture of SO$_2$ = 84.7\% , CO$_2$ = 0\% at a pressure of 0.030 bar and a temperature of 308.8 K compared to various models}
 \end{figure}

\begin{figure}[p]
 \centering 
\includegraphics[width=0.7\textwidth]{./model/results/{{308.55K-0.97bar-25so2-modelComparison}}} 
 \caption{Opacity data  using the 2-3 mm-wavelength system for a mixture of SO$_2$ = 2.6\% , CO$_2$ = 96.9\% at a pressure of 0.970 bar and a temperature of 308.5 K compared to various models}
 \end{figure}

\begin{figure}[p]
 \centering 
\includegraphics[width=0.7\textwidth]{./model/results/{{308.65K-1.995bar-25so2-modelComparison}}} 
 \caption{Opacity data  using the 2-3 mm-wavelength system for a mixture of SO$_2$ = 1.3\% , CO$_2$ = 98.5\% at a pressure of 1.995 bar and a temperature of 308.6 K compared to various models}
 \end{figure}

\begin{figure}[p]
 \centering 
\includegraphics[width=0.7\textwidth]{./model/results/{{307.55K-0.116bar-98so2-modelComparison}}} 
 \caption{Opacity data  using the 3-4 mm-wavelength system for a mixture of SO$_2$ = 84.7\% , CO$_2$ = 0\% at a pressure of 0.116 bar and a temperature of 307.5 K compared to various models}
 \end{figure}

\begin{figure}[p]
 \centering 
\includegraphics[width=0.7\textwidth]{./model/results/{{307.25K-0.9429bar-98so2-modelComparison}}} 
 \caption{Opacity data  using the 3-4 mm-wavelength system for a mixture of SO$_2$ = 10.4\% , CO$_2$ = 87.7\% at a pressure of 0.943 bar and a temperature of 307.2 K compared to various models}
 \end{figure}

\begin{figure}[p]
 \centering 
\includegraphics[width=0.7\textwidth]{./model/results/{{307.25K-1.9869bar-98so2-modelComparison}}} 
 \caption{Opacity data  using the 3-4 mm-wavelength system for a mixture of SO$_2$ = 4.9\% and CO$_2$ = 94.2\% at a pressure of 1.987 bar and a temperature of 307.2 K compared to various models}
 \end{figure}

\begin{figure}[p]
 \centering 
\includegraphics[width=0.7\textwidth]{./model/results/{{344.45K-0.09bar-76so2-modelComparison}}} 
 \caption{Opacity data  using the 2-3 mm-wavelength system for a mixture of SO$_2$ = 84.7\% and CO$_2$ = 0\% at a pressure of 0.090 bar and a temperature of 344.4 K compared to various models}
 \end{figure}

\begin{figure}[p]
 \centering 
\includegraphics[width=0.7\textwidth]{./model/results/{{344.65K-0.923bar-76so2-modelComparison}}} 
 \caption{Opacity data  using the 2-3 mm-wavelength system for a mixture of SO$_2$ = 8.3\% and CO$_2$ = 90.2\% at a pressure of 0.923 bar and a temperature of 344.6 K compared to various models}
 \end{figure}

\begin{figure}[p]
 \centering 
\includegraphics[width=0.7\textwidth]{./model/results/{{343.95K-1.967bar-76so2-modelComparison}}} 
 \caption{Opacity data  using the 2-3 mm-wavelength system for a mixture of SO$_2$ = 3.9\% and CO$_2$ = 95.4\% at a pressure of 1.967 bar and a temperature of 343.9 K compared to various models}
 \end{figure}

\begin{figure}[p]
 \centering 
\includegraphics[width=0.7\textwidth]{./model/results/{{344.35K-0.033bar-28so2-modelComparison}}} 
 \caption{Opacity data  using the 2-3 mm-wavelength system for a mixture of SO$_2$ = 84.7\% and CO$_2$ = 0\% at a pressure of 0.033 bar and a temperature of 344.3 K compared to various models}
 \end{figure}

\begin{figure}[p]
 \centering 
\includegraphics[width=0.7\textwidth]{./model/results/{{344.55K-0.944bar-28so2-modelComparison}}} 
 \caption{Opacity data  using the 2-3 mm-wavelength system for a mixture of SO$_2$ = 3\% and CO$_2$ = 96.5\% at a pressure of 0.944 bar and a temperature of 344.5 K compared to various models}
 \end{figure}

\begin{figure}[p]
 \centering 
\includegraphics[width=0.7\textwidth]{./model/results/{{344.45K-2.007bar-28so2-modelComparison}}} 
 \caption{Opacity data  using the 2-3 mm-wavelength system for a mixture of SO$_2$ = 1.4\% and CO$_2$ = 98.4\% at a pressure of 2.007 bar and a temperature of 344.4 K compared to various models}
 \end{figure}

\begin{figure}[p]
 \centering 
\includegraphics[width=0.7\textwidth]{./model/results/{{343.65K-0.101bar-86so2-modelComparison}}} 
 \caption{Opacity data  using the 3-4 mm-wavelength system for a mixture of SO$_2$ = 84.7\% and CO$_2$ = 0\% at a pressure of 0.101 bar and a temperature of 343.6 K compared to various models}
 \end{figure}

\begin{figure}[p]
 \centering 
\includegraphics[width=0.7\textwidth]{./model/results/{{343.25K-0.936bar-86so2-modelComparison}}} 
 \caption{Opacity data  using the 3-4 mm-wavelength system for a mixture of SO$_2$ = 9.1\% and CO$_2$ = 89.2\% at a pressure of 0.936 bar and a temperature of 343.2 K compared to various models}
 \end{figure}

\begin{figure}[p]
 \centering 
\includegraphics[width=0.7\textwidth]{./model/results/{{342.95K-2.016bar-86so2-modelComparison}}} 
 \caption{Opacity data  using the 3-4 mm-wavelength system for a mixture of SO$_2$ = 4.2\% and CO$_2$ = 95\% at a pressure of 2.016 bar and a temperature of 342.9 K compared to various models}
 \end{figure}

\begin{figure}[p]
 \centering 
\includegraphics[width=0.7\textwidth]{./model/results/{{343.15K-0.06bar-51so2-modelComparison}}} 
 \caption{Opacity data  using the 3-4 mm-wavelength system for a mixture of SO$_2$ = 84.7\% and CO$_2$ = 0\% at a pressure of 0.060 bar and a temperature of 343.1 K compared to various models}
 \end{figure}

\begin{figure}[p]
 \centering 
\includegraphics[width=0.7\textwidth]{./model/results/{{343.65K-0.927bar-51so2-modelComparison}}} 
 \caption{Opacity data  using the 3-4 mm-wavelength system for a mixture of SO$_2$ = 5.5\% and CO$_2$ = 93.5\% at a pressure of 0.927 bar and a temperature of 343.6 K compared to various models}
 \end{figure}

\begin{figure}[p]
 \centering 
\includegraphics[width=0.7\textwidth]{./model/results/{{343.95K-2.004bar-51so2-modelComparison}}} 
 \caption{Opacity data  using the 3-4 mm-wavelength system for a mixture of SO$_2$ = 2.5\% and CO$_2$ = 97\% at a pressure of 2.004 bar and a temperature of 343.9 K compared to various models}
 \end{figure}



\clearpage

\section{Suggested Model}
Results indicate that the models for the centimeter- and millimeter-wavelength opacity from SO$_2$ in a CO$_2$ atmosphere by Suleiman et al. \cite{Suleiman-1996} and Fahd and Steffes \cite{Fahd-1991} are both valid over the entire centimeter-and millimeter-wavelength range under simulated conditions for the upper atmosphere of Venus. Based on the percentage of data consistent with the models, we suggest the model from Fahd and Steffes \cite{Fahd-thesis}, but using the updated line catalog from Picket et al. \cite{Pickett-1998}. This model employs the Van Vleck-Weisskopf lineshape, and was developed from measurements of SO$_2$/CO$_2$ mixtures conducted at room temperature. As per their paper, we employ only the rotational line catalog to compute opacity. (JPL spectral line catalog, Pickett et al., \cite{Pickett-1998}). While both models perform well, the Fahd and Steffes \cite{Fahd-1991}  model appears to provide a slightly better fit to the overall data set. 

It should also be noted that because both models were derived from measurements conducted at pressures of 6 bars or less, no allowance for the compressibility of CO$_2$ is included in these models. When performing the best-fit analysis of high-pressure data \cite{Steffes-2015}, a correction factor for compressibility was computed and entered into the models (by simply dividing the measured partial pressure of CO$_2$ by the compressibility, Z). 


\chapter{Summary and Conclusions}
\section{Significant Results}
\section{Application to Venus Observations}
\section{Suggestions for Future Work}
Venus RTM
%\section{Proposed Research}
The objective of the proposed research is to advance the knowledge of the millimeter-wavelength properties of gaseous sulfur dioxide under Venus conditions. As part of the proposed research, extensive laboratory measurements of the W-band and F-band properties of sulfur dioxide under simulated upper Venus atmosphere are ongoing. 

Upon completion of the laboratory measurements, efforts toward developing a unified model to estimate the centimeter and millimeter-wavelength opacity spectra of sulfur dioxide at various pressures, temperatures, and mixing ratios will be developed. 

\subsection{Significance of this Work}
The primary objective of this millimeter-wavelength research is to better understand properties of gaseous sulfur dioxide under Venus conditions. The laboratory measurements will help create a model that accurately estimates the opacity of sulfur dioxide in a carbon dioxide atmosphere at any temperature or pressure. The new model will provide a unified opacity model for sulfur dioxide at the centimeter and millimeter-wavelengths. This model can be used for accurate retrievals of sulfur dioxide from ground-based and spacecraft-based radio observations. 
\subsection{Planned Work}

Two major activities support this thesis: (1) laboratory measurements of 2-4 millimeter-wavelength properties of sulfur dioxide in a carbon dioxide atmosphere at two temperatures, 308 K and 348 K with pressures up to two bars and (2) the development of a model that best estimates sulfur dioxide's millimeter-wavelength absorption properties in Venus's upper troposphere. Both will be completed by Fall Semester 2014.


\subsubsection{Laboratory Measurements}

The ongoing millimeter-wavelength measurements  focus on characterizing the opacity of sulfur dioxide in a carbon dioxide atmosphere temperatures of 308 K and 348 K. These will be the first precision measurements done of sulfur dioxide's absorption properties at millimeter-wavelengths. Previous work by Fahd \cite{fahd-so2} included a measurement done at 94.1 GHz but it did not utilize the high precision measurement tools available for this thesis.  

The testing protocol involve laboratory measurements of the opacity of only sulfur dioxide as well as a mixture of sulfur dioxide and carbon dioxide at pressures of 1 bar and 2 bar. Table \ref{tab:testmatrix} shows a testing matrix of the tests to be done. Varying the sulfur dioxide abundance as well as the temperature allows for an accurate model to be created and extrapolated to other pressure and temperature combinations. 

\begin{table}[H]
    \centering
	\begin{tabular}{|c|c|c|c|c|}
	\hline
	Test Number & Gas under test & Pressure & Temperature & Subsystem\\
	\hline
	\hline
	1 & SO$_2$ & 100 mbar & 308 K & W-band\\
	2 & SO$_2$ & 100 mbar & 343 K & W-band\\
	3 & SO$_2$ & 60 mbar & 343 K & W-band\\
	4 & SO$_2$ & 100 mbar & 343 K & F-band\\
	5 & SO$_2$ & 30 mbar & 343 K & F-band\\
	6 & SO$_2$ & 30 mbar & 308 K & F-band\\
	\hline
	\end{tabular}
	\caption{Testing matrix for SO$_2$'s microwave absorption properties at 2-4 millimeter-wavelength.}
    \label{tab:testmatrix}
\end{table}


\subsubsection{Model Development}

Completed and planned centimeter- \cite{so2-cent-lab} \cite{so2-cent-model} and millimeter-wavelength measurements of the opacity of sulfur dioxide under Venus atmospheric conditions will be used to create a new model that accurately characterizes the centimeter- and millimeter-wavelength properties of sulfur dioxide. After the completion of the millimeter-wavelength laboratory measurements, both centimeter- and millimeter-wavelength data will be put into an optimization algorithm to create the best model estimate of the absorptivity of sulfur dioxide under Venus conditions. 

\subsection{Facilities Required}
All of the facilities required for this work currently exist in the Planetary Atmospheres Laboratory at The Georgia Institute of Technology. The facilities include millimeter-wavelength test equipment, a Fabry-Perot resonator, a temperature chamber, pressure and temperature gauges, and a computer running Matlab. Additional resources that are required and available are gas cylinders of carbon dioxide and sulfur dioxide. Computing resources necessary for model development are available at the Planetary Atmospheres Laboratory.

\subsection{Milestone}

\subsubsection*{Completed Work}
\begin{itemize}
\item Completed setup of millimeter-wavelength system: January 2014
\item Completed 3-4 millimeter-wavelength opacity of SO$_2$ under simulated Venus conditions: February 2014
\item Completed 2-3 millimeter-wavelength opacity of SO$_2$ under simulated Venus conditions: April 2014
\item Diagnosed issue with SO$_2$ pressure regulator and modified data accordingly: April 2014
\item Submit thesis proposal: April 2014
\end{itemize}
\enlargethispage{\baselineskip}
\subsubsection*{Remaining Work}
\begin{itemize}
\item Verify constituent inventory in SO$_2$ bottle and modify data accordingly: April 2014
\item Develop consistent model for opacity of SO$_2$ incorporating the millimeter-wavelength measurements developed as part of this work and the centimeter-wavelength measurements made previously: June 2014
\item Submit journal paper detailing the millimeter-wavelength measurements and the opacity model: November 2014
\item Submit Master's Thesis: December 2014
\end{itemize}

\newpage

\bibliographystyle{elsart-num-sort} 
\bibliography{refs}
\end{document}
