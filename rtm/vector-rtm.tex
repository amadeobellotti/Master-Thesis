\section{Vector Radiative Transfer}
A typical method of radiative transfer modeling is to iterate through each layer and calculate the layer's RTM parameters and temperature. 
While computing an RTM this way is easier to understand, it is extremely inefficient. The following section describes a more efficient way of computing a radiative transfer model.
\subsection{Thermo-Chemical Model (TCM)}
The first step of a vector radiative transfer model is to compute the TCM for the Venus atmosphere. The TCM is dependent on the altitude vector \textbf{a} whose size is $N\times 1$ where N is the number of layers in the altitude and the vector of all interested frequencies \textbf{F} which is $1 \times M$ with M being the number of interested frequencies. \textbf{a} is defined as
\begin{equation}
a_i = iz_{step}
\end{equation}
where $a_i$ is the $i^{th}$ element in the vector and $z_{step}$ is the distance between each atmospheric layer. 

The TCM for Venus requires a latitude of observation. This is due to the latitudinal variations of the temperature-pressure profiles of the planet. 
Using the altitude vector, \textbf{a}, it is possible to calculate the T-P profiles of the interested atmospheric layers using a one dimension interpolation of the T-P profiles as reported by the Pionner-Venus sounder and north probes. 
The temperature and pressure vectors are \textbf{T} and \textbf{P} respectively. Using the \textbf{T} and \textbf{P}, it will be possible to create a vector for all constituents mixing ratio $\textbf{Q}_{c}$, with $c$ being the constituent of interest. The refractive index vector \textbf{N} can be calculated using the same methods as $\textbf{Q}_{c}$. The vectors \textbf{T}, \textbf{P}, $\textbf{Q}_{c}$, and \textbf{N} are of size $N\times 1$.

\subsection{Absorption Matrix}
The absorption matrix \textbf{A} needs to be calculated. This is done by
\begin{equation}
A_{i,f} = \sum_{constituents} \alpha_{i,c}(\textbf{F}(f))
\end{equation}
where $A_i$ is the $i^{th}$ element in the vector and $\alpha_{i,c}(f)$ is the absorption of the constituent $c$ at the $i^{th}$ layer in the atmosphere with a frequency of $\textbf{F}(f)$. \textbf{A} is of size $N \times M$.

\subsection{Ray-Tracing}
In this method, Ray-Tracing is still done iteratively, but in this case we start with a distance vector \textbf{d} of size $N \times 1$ such that all elements in the vector are zero,
\begin{equation*}
\textbf{d} = \vec{\textbf{0}}
\end{equation*}
and for every $t$, distance the ray traveled in a layer, calculated in the Ray-tracing algorithm the vector \textbf{d} is updated using
\begin{equation}
\textbf{d}_i = \textbf{d}_i + t 
\end{equation}
This keeps track of the total distance spent in each layer.

\subsection{Radiative Transfer Model}

Several variables are calculated in this RTM. The first is the opacity matrix, $\vec{\tau}$, which is defined as
\begin{equation}
\tau_{i,j} = \alpha_{i,j} \times \textbf{d}_i
\end{equation}
where $\alpha$ is the opacity at layer $i$ at frequency $j$, and $\textbf{d}$ is the distance the ray travels through layer $i$

Using the opacity matrix it is possible to calculate the weighting matrix for the upwelling and downwelling of the atmosphere, $\textbf{W}_{up}$ and $\textbf{W}_{down}$ respectively, using the following
\begin{equation}
\textbf{W}_{up_{i,j}} = (1-e^{-\tau_{i,j}})e^{\left(-\sum_{l=i+1}^N \tau_{l,j}\right)}
\end{equation}
\begin{equation}
\textbf{W}_{down_{i,j}} = (1-e^{-\tau_{i,j}})e^{\left(-\sum_{l=1}^{i-1} \tau_{l,j}\right)} e^{\left(-\sum_{l=1}^{N} \tau_{l,j}\right)} (1- \epsilon(\theta))
\end{equation}
where $i$ is again each layer of the atmosphere, $j$ is each frequency of interest and $\epsilon(\theta)$ is the surface emissivity. $\textbf{W}_{up}$ calculates the attenuation of the current layer and every layer above it. $\textbf{W}_{down}$ calculates the attenuation from the current layer towards the surface and back through the entire atmosphere. 

These weighting vectors along with the temperature vector, \textbf{T}, gives the expected temperature brightness through
\begin{equation}
\begin{split}
\textbf{Tb}_j 	& = T_{surf}\cdot \epsilon(\theta)\cdot e^{\left(-\sum_{l=1}^{N} \tau_{l,j}\right)} + T_{cmb}\cdot (1-\epsilon(\theta))\cdot e^{\left(-2\sum_{l=1}^{N} \tau_{l,j}\right)} \\
				& + \sum_{i=1}^N \textbf{T}_i \cdot \textbf{W}_{up_{i,j}} + \sum_{i=1}^N \textbf{T}_i \cdot \textbf{W}_{down_{i,j}}
\end{split}
\end{equation}
where the first term is the temperature at the surface multiplied by the emissivity and attenuated by the atmosphere. The second term is the cosmic microwave background (2.7K) multiplied by the reflectivity of the planet then attenuated by the atmosphere twice (down and back up). The third term is the upwelling of the atmosphere which is the temperature at each level multiplied by the upwelling weighting matrix described previously. The final term is the downwelling of the atmosphere which again is the temperature at each level multiplied by the downwelling weighting matrix defined previously. 
